\chapter*{Záver}
\addcontentsline{toc}{chapter}{Záver}

Jedným z dôvodov, prečo ľudia sledujú šport je ten, že šport je nepredvídateľný. Na druhej strane, ľudstvo sa učí od nepamäti získavať kontrolu nad nepredvídateľným a ak sa to nepodarí, tak aspoň to s nejakou určitosťou predvídať (napríklad zatmenie slnka). Vedieť predvídať športové výsledky by nám zaručilo nielen slávu a pozornosť médií, ale mohlo by nám to pred týmto všetkým zaručiť aj bohatstvo, pretože stávkové kancelárie dávajú možnosť zarobiť na správnom tipovaní (v našom prípade predpovedaní) výsledkov.

Jedným z relatívne nových spôsobov, ktorý sa začína používať na predpovedanie športových výsledkov je strojové učenie. Jedným z najznámejších a momentálne aj naobľúbenejších typov strojového učenia sú neurónové siete. V tejto práci sme sa zamerali na dva druhy neurónových sietí, a to dopredné a rekurentné neurónové siete a ich úspešnosť pri predikcii výsledkov v troch futbalových ligách a na najvyšších tenisových turnajoch medzi najlepšími hráčmi.

Používali sme len dáta, ktoré vieme vyčítať z výsledkov (v prípade tenisu aj z koncoročných rebríčkov), ale nepodarilo sa nám ani priblížiť výsledkom iných autorov publikujúcich v tejto oblasti, ktorý používali aj fakticky nepodložené abstraktné dáta od expertov v oblasti. V prípade futbalu sme dosiahli úspešnosť od 50,54 -- 56,24 \,\%. V prípade tenisu sme dosiahli úspešnosť okolo 65,5\,\%.

Jednou z odlišností tejto práce od ostatných v oblasti bolo zameranie sa aj na zápasy bez jasného favorita stávkových kancelárií. Aj napriek tomu, že to bol celkom nový prístup, nepodarilo sa nám dosiahnúť požadované výsledky, ktoré by mohli podnietiť väčší výskum týmto smerom. To ale neznamená, že sa lepšie výsledky dosiahnuť nedajú. Znamená to iba, že sa zo získaných dát ukázaným spôsobom nebude dať vytvoriť neurónová ani rekurentná neurónová sieť, ktorá by dosahovala stabilné výsledky a dokázala by pravidelne a dlhodobo poraziť stávkové kancelárie a vykázať teda nejaký zisk.