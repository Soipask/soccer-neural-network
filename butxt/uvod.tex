\chapter*{Úvod}
\addcontentsline{toc}{chapter}{Úvod}
Šport je súčasťou zábavného priemyslu hlavne pre relatívnu nepredvítateľnosť jeho výsledkov. 
Stať sa môže v podstate čokoľvek. 
Vyhrať môže favorit udalosti alebo osoba/tím, od ktorej sa to vôbec neočakávalo.
Môže začať pršať alebo na ihrisko vbehnúť exhibicionista s kontroverznou myšlienkou.

Táto nepredvídateľnosť podnietila vznik stávkových kancelárií, ktoré na tieto a na rôzne ďalšie udalosti vypisujú kurzy, ktoré v prípade, že tieto udalosti nastanú, zaručia stávkujúcemu výhru.
Ich ziskovosť je založená na vypisovaní kurzov tak, aby boli lákavé pre bežných ľudí.
V podstate sa snažia uhádnuť, s akou pravdepodobnosťou nastane daná udalosť, napríklad predikovať výsledok.
Stávkové kancelárie používajú na tieto odhady nejaké dáta, ale pravdepodobnosti daných udalostí zvykne predpovedať odborník, bookmaker.
Je možné nájsť nejakú množinu dát, na základe ktorej vieme naučiť počítač predikovať výsledky jednotlivých športových udalostí s určitou presnosťou?

\section*{Súvisiace práce}

V minulosti boli použité rôzne metódy na predikciu športových výsledkov.
V~roku 2005 sa o predpoveď 6 rôznych udalostí týkajúcich sa austrálskej kriketovej ligy a AFL, ligy v austrálskom futbale, pokúsil Bailey \citep{related:bailey}. 
Na austrálsky futbal použil data zo zápasov zo 100 sezón odohraných pred rokom 1997 a testoval to na zápasoch od sezóny 1997 do 2003 použitím rôznych modelov lineárnej regresie. 
Dokázal získať presnosť 66.7\%.

V roku 2006 Joseph, Fenton a Neil vyskúšali viaceré druhy strojového učenia na predikciu výsledkov zápasov tímu Tottenham Hotspur F.C. v najvyššej anglickej futbalovej lige, Premier League, v sezónach 1995/1996 a 1996/1997 \citep{related:joseph}.
To znamená, že pracovali s datasetom o veľkosti 76 zápasov, z ktorého časť delili na trénovacie a časť na testovacie data. 
Použité metódy zahŕňali expertmi konštruované bayesovské siete, naivný bayesovký klasifikátor, rozhodovacie stromy a k-NN (k nearest neighbours clustering). 
Použili pri tom 30 príznakov, ale 28 sa viazalo iba na to, či daný hráč nastúpil od začiatku na daný zápas alebo nie, zvyšné dva predstavovali silu súpera a miesto zápasu (či hral predikovaný tím na domácom štadióne alebo nie).
V tomto prípade dosiahli bayesovské siete úspešnosť niečo vyše 59\%, zvyšné metódy sa pohybovali v rozmedzí 30 -- 38\% pri disjunktných testovacích a trénovacích datach.

V roku 2011 sa dvojica Hucaljuk a Rakipovi{\'c} zameriavala na výber príznakov pri predikcii výsledkov futbalovej Ligy majstrov \citep{related:hucaljuk}. 
Pracovali s datami z 96 zápasov, ktoré manuálne ohodnotili podľa 30 príznakov.
Vybrané príznaky predstavovali formu oboch tímov v posledných 6 zápasoch, výsledok posledného vzájomného zápasu týchto dvoch tímov, postavenie v rebríčku, počet zranených hráčov a priemerný počet strelených a inkasovaných gólov.
Neskôr zúžili počet príznakov na 20 a na novovzniknutý dataset bolo aplikovaných 6 rôznych metód strojového učenia, menovite: naivný bayesovský klasifikátor, bayesovské siete, LogitBoost, k-NN, random forest a neurónové siete. Najvyššia dosiahnutá úspešnosť bola 68\%, ktorú dosiahli použitím neurónových sietí.

V roku 2014 použili Igiri a Nwachukwu nástroj zvaný Rapid Miner \citep{related:igiri}. 
Jeho úlohou bolo predikovať výsledky anglickej Premier League. 
Použité techniky boli popredná neurónová sieť a lineárna regresia. 
Neurónová sieť dosiahla úspešnosti 85\%, lineárna regresia 93\%. 
Je potrebné dodať, že neurónová sieť predpovedala všetky typy výsledkov (výhra domácich, prehra, remíza), zatiaľ čo regresia predpovedala len zápasy, ktoré sa v konečnom dôsledku skončili výhrou alebo prehrou domáceho celku, takže celková úspešnosť bola o niečo nižšia. 
Autori dodali, že ak sa predpokladá, že zápas môže skončiť aj remízou, tak neurónové siete mali lepšie výsledky. 
K predikcii použili rôzne príznaky vrátane kurzov, priemerný počet striel, striel na bránu, rohových kopov, ale aj abstraktnejšie príznaky ako ofenzívna/defenzívna sila mužstva a ohodnotenie sily jednotlivých hráčov a kvality manažéra.

V tom istom roku sa Shin a Gasparyan pokúsili nájsť nové metódy predikcie \citep{related:shin}. 
Navrhli použiť data z videohry FIFA 2015 na pre\-dik\-ciu španielskej La Ligy.
Použitie tohto návrhu odôvodnili tým, že vydavatelia videohier v dnešnej dobe pracujú na tom, aby boli ich hry čo možno najreálnejšie.
To sa hlavne týka športových hier, kde je dôležité, aby sa hodnotenie hráča čo najviac približovalo realite.
FIFA 2015 používa rôzne atribúty na ohodnotenie hráča, ako napríklad zrýchlenie, strely z diaľky alebo reflexy pre post brankára.
Tieto data sa získavajú oveľa jednoduchšie ako z iných zdrojov.
Autori vytvorili dva typy modelov: učenie s učiteľom a bez učiteľa.
Pri učení s učiteľom vytvorili 2 prístupy, reálny prediktor, ktorý využíval reálne data a virtuálny prediktor, ktorý využíval práve data z popísanej videohry.
Obe využívali logistickú regresiu a metódu podporných vektorov.
Reálny prediktor dosiahol úspešnosť 75\%, virtuálny 80\%, čo podľa autorov dokazuje, že data získané z videohier sa dajú používať aj v reálnom svete.
Učenie bez učiteľa analyzovalo stratégie tímov podľa typov hráčov, ktorí sú v danom tíme pomocou k-means clusteringu. Zistili, že lepšie tímy zvyknú mať útočnejšie stratégie a slabšie tímy dokážu uhrať lepšie výsledky proti silnejším tímom, ak majú defenzívnejšiu stratégiu.

Taktiež v roku 2014 sa v Iráne skupina výskumníkov pokúsila predpovedať výsledky posledného kola najvyššej iránskej futbalovej ligy IPL zo sezóny 2013/2014 \citep{related:iran}.
Pred posledným kolom nebolo nič rozhodnuté a väčšina z 16 tímov v lige bojovala o lepšie umiestnenie, 5 tímov bojovalo dokonca o titul.
Pri rovnosti bodov záleží vo futbale aj na rozdiele v počte strelených a inkasovaných gólov. 
Kvôli vyrovnanosti ligy sa títo výskumníci pokúsili predikovať presné výsledky, teda presný počet gólov strelených domácim i hosťujúcim mužstvom vo všetkých 8 zápasoch.
Získali informácie z viac ako 1800 predchádzajúcich zápasov ligy a k predikcii použili rôzne príznaky vrátane počtu získaných bodov počas sezóny, počtu získaných bodov v posledných 4 zápasoch a kvality súpera počas posledných 4 zápasov, spolu aj s identifikačnými kódmi jednotlivých tímov a kolom, v ktorom sa daný zápas odohral.
Celkovo použili 10 príznakov, na predikciu použili neurónovú sieť.
Vo výsledku správne predpovedali víťaza ligy, vzájomné poradie medzi štyrmi z 5 tímov, ktoré bojovali o víťazstvo v lige a presné poradie posledných 5 tímov v tabuľke.

V roku 2016 vyskúšali logistickú regresiu na predikciu výsledkov futbalovej Premier League výskumníci z tímu Prasetia \citep{related:prasetio}. 
Stavali na výsledkoch svojich predchodcov a vybrali 4 príznaky, ktoré hrali v predchádzajúcich prácach najväčšiu rolu, konkrétne ohodnotenia pre obranu a útok, pre domácich aj hostí.
Dosiahli úspešnosti v najlepšom prípade 69,5\%.
\section*{Prínos práce}
V tejto práci budeme predikovať futbal a tenis pomocou popredných a rekurentných neurónových sietí.
Tenis nie je predikovaný v žiadnej z prác spomínaných v predchádzajúcej sekcii.
Futbal je síce predikovaný, ale ani raz štýlom, aký bude prezentovaný v tejto práci.

Väčšina prác má oveľa menšiu trénovaciu vzorku pre siete. 
Pre túto prácu boli použité informácie z viac ako 5000 futbalových zápasov, z toho trénovacia množina tvorila viac ako 3000 vstupov pre každú ligu.
Pre tenis obsahuje dátaset viac ako 6200 riadkov a je vytvorený z informácií z viac ako 55000 zápasov.

Ďalší atribut, ktorým sa táto práca odlišuje od ostatných predstavuje použité dáta. V tejto práci budú použité výhradne výsledky a prostredie zápasov, z ktorých sú následne kalkulované ostatné informácie.
Nebudú použité abstraktné dáta ako sila hráčov alebo tímu ani ohodnotenia žiadnych hráčov ako ani dáta o počte rohových kopov, žltých kariet, es alebo nevynútených chýb. 
V tomto ohľade je najpodobnejšia práca od iránskych výskumníkov \citep{related:iran}, ale aj tam sú značné rozdiely v použití dát.

Žiadna z vyššie spomínaných prác nepoužíva ako jednu z metód vyhodnocovania sietí kurzy stávkových kancelárií.
V tejto práci nás hlavne zaujímajú zápasy, v ktorých ani jeden z tímov nie je favoritom z hľadiska kurzov stávkových kancelárií.
Vyhodnocovať teda budeme celkovú úspešnosť siete; zisk, ktorý by sme dosiahli pri stávkovaní na všetky zápasy a zisk, ktorý by sme dosiahli stávkovaním výhradne na zápasy, v ktorých nie je jasný favorit.
