%%% Hlavní soubor. Zde se definují základní parametry a odkazuje se na ostatní části. %%%

%% Verze pro jednostranný tisk:
% Okraje: levý 40mm, pravý 25mm, horní a dolní 25mm
% (ale pozor, LaTeX si sám přidává 1in)
\documentclass[12pt,a4paper]{report}
\setlength\textwidth{145mm}
\setlength\textheight{247mm}
\setlength\oddsidemargin{15mm}
\setlength\evensidemargin{15mm}
\setlength\topmargin{0mm}
\setlength\headsep{0mm}
\setlength\headheight{0mm}
% \openright zařídí, aby následující text začínal na pravé straně knihy
\let\openright=\clearpage

%% Pokud tiskneme oboustranně:
% \documentclass[12pt,a4paper,twoside,openright]{report}
% \setlength\textwidth{145mm}
% \setlength\textheight{247mm}
% \setlength\oddsidemargin{14.2mm}
% \setlength\evensidemargin{0mm}
% \setlength\topmargin{0mm}
% \setlength\headsep{0mm}
% \setlength\headheight{0mm}
% \let\openright=\cleardoublepage

\usepackage[usenames]{xcolor}  % barevná sazba

%% Vytváříme PDF/A-2u
\usepackage[a-2u]{pdfx}

%% Přepneme na českou sazbu a fonty Latin Modern
\usepackage[slovak]{babel}
\usepackage{lmodern}
\usepackage[T1]{fontenc}
\usepackage{textcomp}

%% Použité kódování znaků: obvykle latin2, cp1250 nebo utf8:
\usepackage[utf8]{inputenc}

%%% Další užitečné balíčky (jsou součástí běžných distribucí LaTeXu)
\usepackage{amsmath}        % rozšíření pro sazbu matematiky
\usepackage{amsfonts}       % matematické fonty
\usepackage{amsthm}         % sazba vět, definic apod.
\usepackage{bbding}         % balíček s nejrůznějšími symboly
			    % (čtverečky, hvězdičky, tužtičky, nůžtičky, ...)
\usepackage{bm}             % tučné symboly (příkaz \bm)
\usepackage{graphicx}       % vkládání obrázků
\usepackage{fancyvrb}       % vylepšené prostředí pro strojové písmo
\usepackage{indentfirst}    % zavede odsazení 1. odstavce kapitoly
\usepackage{natbib}         % zajištuje možnost odkazovat na literaturu
			    % stylem AUTOR (ROK), resp. AUTOR [ČÍSLO]
\usepackage[nottoc]{tocbibind} % zajistí přidání seznamu literatury,
                            % obrázků a tabulek do obsahu
\usepackage{icomma}         % inteligetní čárka v matematickém módu
\usepackage{dcolumn}        % lepší zarovnání sloupců v tabulkách
\usepackage{booktabs}       % lepší vodorovné linky v tabulkách
\usepackage{paralist}       % lepší enumerate a itemize


\usepackage{listings}
\usepackage{color}

\usepackage{multicol}

%%% Údaje o práci

% Název práce v jazyce práce (přesně podle zadání)
\def\NazevPrace{Predikce sportovních utkání pomocí neuronových sítí}

% Název práce v angličtině
\def\NazevPraceEN{Prediction of sports results using neural networks}

% Jméno autora
\def\AutorPrace{Daniel Šipoš}

% Rok odevzdání
\def\RokOdevzdani{2019}

% Název katedry nebo ústavu, kde byla práce oficiálně zadána
% (dle Organizační struktury MFF UK, případně plný název pracoviště mimo MFF)
\def\Katedra{Katedra softvéru a výuky informatiky}
\def\KatedraEN{Department of Software and Computer Science Education}

% Jedná se o katedru (department) nebo o ústav (institute)?
\def\TypPracoviste{Katedra}
\def\TypPracovisteEN{Department}

% Vedoucí práce: Jméno a příjmení s~tituly
\def\Vedouci{Mgr. David Kuboň}

% Pracoviště vedoucího (opět dle Organizační struktury MFF)
\def\KatedraVedouciho{Katedra softvéru a výuky informatiky}
\def\KatedraVedoucihoEN{Department of Software and Computer Science Education}

% Studijní program a obor
\def\StudijniProgram{Informatika}
\def\StudijniObor{Obecná~informatika}

% Nepovinné poděkování (vedoucímu práce, konzultantovi, tomu, kdo
% zapůjčil software, literaturu apod.)
\def\Podekovani{%
Poděkování.
}

\iffalse

Stávkové kancelárie každoročne zarábajú milióny eur na~hazarde zvanom stávkovanie. Často na~to používajú veľké množstvá špecifických informácií a~pre významnejšie udalosti dokonca špeciálnych ľudí, bookmakerov, ktorí osobne sledujú podobné udalosti, či~už ide o~známe futbalové ligy, veľké cyklistické podujatia alebo tradičné odhalenie mena novorodeného nasledovníka v~kráľovskom rode. Táto práca sa zameriava na~otázku, či~dokážeme pomocou neurónových sietí dlhodobo zarábať stávkovaním na~výsledky športových zápasov. Stávky sú realizované na~známych futbalových ligách. Taktiež sú v~tejto práci prezentované dva druhy neurónových sietí, a~to klasická a rekurentná neurónová sieť, tieto dva druhy medzi sebou porovnáme z~viacerých uhlov.

\fi

% Abstrakt (doporučený rozsah cca 80-200 slov; nejedná se o zadání práce)
\def\Abstrakt{%
Futbal a~tenis patria k~najpopulárnejším športom na~tejto planéte. Platí to hlavne vďaka jednoduchosti pravidiel a~nenáročnosti na~vybavenie. Obe športy môže hrať prakticky ktokoľvek. Sú ale~ľudia, ktorým tieto športy idú lepšie ako~ostatným, takzvaní profesionáli. Títo profesionáli potom chodia po~rôznych turnajoch, resp. ligách, kde hrajú zápasy, aby~sa ukázalo, kto je najlepší. Ľudská chamtivosť na~jednej strane a závislosť na~hazarde na~druhej podporujú vznik rôznych spoločností, stávkových kancelárií, ktoré umožňujú tipovať výsledky týchto zápasov za~peniaze. Táto práca sa zameriava na~predpovedanie takýchto výsledkov futbalových líg a~tenisových turnajov pomocou dvoch mierne odlišných druhov neurónových~sietí, porovnanie jednotlivých predpovedných modelov a~taktiež porovnanie predvídateľnosti futbalu a~tenisu.
}

\def\AbstraktEN{%
Abstract.
}

% 3 až 5 klíčových slov (doporučeno), každé uzavřeno ve složených závorkách
\def\KlicovaSlova{%
{neurónová sieť} {rekurentná neurónová sieť} {športové stávkovanie} {športové kurzy} {futbal} {tenis}
}
\def\KlicovaSlovaEN{%
{neural network} {recurrent neural network} {football} {tenis} {sport betting} {sport odds} 
}

%% Balíček hyperref, kterým jdou vyrábět klikací odkazy v PDF,
%% ale hlavně ho používáme k uložení metadat do PDF (včetně obsahu).
%% Většinu nastavítek přednastaví balíček pdfx.
\hypersetup{unicode}
\hypersetup{breaklinks=true}

%% Definice různých užitečných maker (viz popis uvnitř souboru)
\include{makra}

%% Titulní strana a různé povinné informační strany
\begin{document}
\include{titulka}

%%% Strana s automaticky generovaným obsahem bakalářské práce

\tableofcontents

%%% Jednotlivé kapitoly práce jsou pro přehlednost uloženy v samostatných souborech
\chapter*{Úvod}
\addcontentsline{toc}{chapter}{Úvod}
Šport je súčasťou zábavného priemyslu hlavne pre relatívnu nepredvítateľnosť jeho výsledkov. 
Stať sa môže v podstate čokoľvek. 
Vyhrať môže favorit udalosti alebo osoba/tím, od ktorej sa to vôbec neočakávalo.
Môže začať pršať alebo na ihrisko vbehnúť exhibicionista s kontroverznou myšlienkou.

Táto nepredvídateľnosť podnietila vznik stávkových kancelárií, ktoré na tieto a na rôzne ďalšie udalosti vypisujú kurzy, ktoré v prípade, že tieto udalosti nastanú, zaručia stávkujúcemu výhru.
Ich ziskovosť je založená na vypisovaní kurzov tak,, aby boli lákavé pre bežných ľudí.
V podstate sa snažia uhádnuť, s akou pravdepodobnosťou nastane daná udalosť, napríklad predikovať výsledok.
Stávkové kancelárie používajú na tieto odhady nejaké data, ale pravdepodobnosti daných udalostí zvykne predpovedať odborník, bookmaker.
Je možné nájsť nejakú množinu dát, na základe ktorej vieme naučiť počítač predikovať výsledky jednotlivých športových udalostí s určitou presnosťou?

\section*{Súvisiace práce}

V minulosti boli použité rôzne metódy na predikciu športových výsledkov.
V~roku 2005 sa o predpoveď 6 rôznych udalostí týkajúcich sa austrálskej kriketovej ligy a AFL, ligy v austrálskom futbale, pokúsil Bailey \citep{related:bailey}. 
Na austrálsky futbal použil dáta zo zápasov zo 100 sezón odohraných pred rokom 1997 a testoval to na zápasoch od sezóny 1997 do 2003 použitím rôznych modelov lineárnej regresie. 
Dokázal získať presnosť 66.7\%.

V roku 2006 Joseph, Fenton a Neil vyskúšali viaceré druhy strojového učenia na predikciu výsledkov zápasov tímu Tottenham Hotspur F.C. v najvyššej anglickej futbalovej lige, Premier League, v sezónach 1995/1996 a 1996/1997 \citep{related:joseph}.
To znamená, že pracovali s dátasetom o veľkosti 76 zápasov, z ktorého časť delili na trénovacie a časť na testovacie dáta. 
Použité metódy zahŕňali expertmi konštruované bayesovské siete, naivný bayesovký klasifikátor, rozhodovacie stromy a k-NN (k nearest neighbours clustering). 
Použili pri tom 30 príznakov, ale 28 sa viazalo iba na to, či daný hráč nastúpil od začiatku na daný zápas alebo nie, zvyšné dva predstavovali silu súpera a miesto zápasu (či hral predikovaný tím na domácom štadióne alebo nie).
V tomto prípade dosiahli bayesovské siete úspešnosť niečo vyše 59\%, zvyšné metódy sa pohybovali v rozmedzí 30 -- 38\% pri disjunktných testovacích a trénovacích dátach.

V roku 2011 sa dvojica Hucaljuk a Rakipovi{\'c} zameriavala na výber príznakov pri predikcii výsledkov futbalovej Ligy majstrov \citep{related:hucaljuk}. 
Pracovali s dátami z 96 zápasov, ktoré manuálne ohodnotili podľa 30 príznakov.
Vybrané príznaky predstavovali formu oboch tímov v posledných 6 zápasoch, výsledok posledného vzájomného zápasu týchto dvoch tímov, postavenie v rebríčku, počet zranených hráčov a priemerný počet strelených a inkasovaných gólov.
Neskôr zúžili počet príznakov na 20 a na novovzniknutý dátaset bolo aplikovaných 6 rôznych metód strojového učenia, menovite: naivný bayesovský klasifikátor, bayesovské siete, LogitBoost, k-NN, random forest a neurónové siete. Najvyššia dosiahnutá úspešnosť bola 68\%, ktorú dosiahli použitím neurónových sietí.

V roku 2014 použili Igiri a Nwachukwu nástroj zvaný Rapid Miner \citep{related:igiri}. 
Jeho úlohou bolo predikovať výsledky anglickej Premier League. 
Použité techniky boli popredná neurónová sieť a lineárna regresia. 
Neurónová sieť dosiahla úspešnosti 85\%, lineárna regresia 93\%. 
Je potrebné dodať, že neurónová sieť predpovedala všetky typy výsledkov (výhra domácich, prehra, remíza), zatiaľ čo regresia predpovedala len zápasy, ktoré sa v konečnom dôsledku skončili výhrou alebo prehrou domáceho celku, takže celková úspešnosť bola o niečo nižšia. 
Autori dodali, že ak sa predpokladá, že zápas môže skončiť aj remízou, tak neurónové siete mali lepšie výsledky. 
K predikcii použili rôzne príznaky vrátane kurzov, priemerný počet striel, striel na bránu, rohových kopov, ale aj abstraktnejšie príznaky ako ofenzívna/defenzívna sila mužstva a ohodnotenie sily jednotlivých hráčov a kvality manažéra.

V tom istom roku sa Shin a Gasparyan pokúsili nájsť nové metódy predikcie \citep{related:shin}. 
Navrhli použiť dáta z videohry FIFA 2015 na pre\-dik\-ciu španielskej La Ligy.
Použitie tohto návrhu odôvodnili tým, že vydavatelia videohier v dnešnej dobe pracujú na tom, aby boli ich hry čo možno najreálnejšie.
To sa hlavne týka športových hier, kde je dôležité, aby sa hodnotenie hráča čo najviac približovalo realite.
FIFA 2015 používa rôzne atribúty na ohodnotenie hráča, ako napríklad zrýchlenie, strely z diaľky alebo reflexy pre post brankára.
Tieto dáta sa získavajú oveľa jednoduchšie ako z iných zdrojov.
Autori vytvorili dva typy modelov: učenie s učiteľom a bez učiteľa.
Pri učení s učiteľom vytvorili 2 prístupy, reálny prediktor, ktorý využíval reálne dáta a virtuálny prediktor, ktorý využíval práve dáta z popísanej videohry.
Obe využívali logistickú regresiu a metódu podporných vektorov.
Reálny prediktor dosiahol úspešnosť 75\%, virtuálny 80\%, čo podľa autorov dokazuje, že dáta získané z videohier sa dajú používať aj v reálnom svete.
Učenie bez učiteľa analyzovalo stratégie tímov podľa typov hráčov, ktorí sú v danom tíme pomocou k-means clusteringu. Zistili, že lepšie tímy zvyknú mať útočnejšie stratégie a slabšie tímy dokážu uhrať lepšie výsledky proti silnejším tímom, ak majú defenzívnejšiu stratégiu.

Taktiež v roku 2014 sa v Iráne skupina výskumníkov pokúsila predpovedať výsledky posledného kola najvyššej iránskej futbalovej ligy IPL zo sezóny 2013/2014 \citep{related:iran}.
Pred posledným kolom nebolo nič rozhodnuté a väčšina z 16 tímov v lige bojovala o lepšie umiestnenie, 5 tímov bojovalo dokonca o titul.
Pri rovnosti bodov záleží vo futbale aj na rozdiele v počte strelených a inkasovaných gólov. 
Kvôli vyrovnanosti ligy sa títo výskumníci pokúsili predikovať presné výsledky, teda presný počet gólov strelených domácim i hosťujúcim mužstvom vo všetkých 8 zápasoch.
Získali informácie z viac ako 1800 predchádzajúcich zápasov ligy a k predikcii použili rôzne príznaky vrátane počtu získaných bodov počas sezóny, počtu získaných bodov v posledných 4 zápasoch a kvality súpera počas posledných 4 zápasov, spolu aj s identifikačnými kódmi jednotlivých tímov a kolom, v ktorom sa daný zápas odohral.
Celkovo použili 10 príznakov, na predikciu použili neurónovú sieť.
Vo výsledku správne predpovedali víťaza ligy, vzájomné poradie medzi štyrmi z 5 tímov, ktoré bojovali o víťazstvo v lige a presné poradie posledných 5 tímov v tabuľke.

V roku 2016 vyskúšali logistickú regresiu na predikciu výsledkov futbalovej Premier League výskumníci z tímu Prasetia \citep{related:prasetio}. 
Stavali na výsledkoch svojich predchodcov a vybrali 4 príznaky, ktoré hrali v predchádzajúcich prácach najväčšiu rolu, konkrétne ohodnotenia pre obranu a útok, pre domácich aj hostí.
Dosiahli úspešnosti v najlepšom prípade 69,5\%.
\section*{Prínos práce}
V tejto práci budeme predikovať futbal a tenis pomocou popredných a rekurentných neurónových sietí.
Tenis nie je predikovaný v žiadnej z prác spomínaných v predchádzajúcej sekcii.
Futbal je síce predikovaný, ale ani raz štýlom, aký bude prezentovaný v tejto práci.

Väčšina prác má oveľa menšiu trénovaciu vzorku pre siete. 
Pre túto prácu boli použité informácie z viac ako 5000 futbalových zápasov, z toho trénovacia množina tvorila viac ako 3000 vstupov pre každú ligu.
Pre tenis obsahuje dataset viac ako 6200 riadkov a je vytvorený z informácií z viac ako 55000 zápasov.

Ďalší atribut, ktorým sa táto práca odlišuje od ostatných predstavuje použité data. V tejto práci budú použité výhradne výsledky a prostredie zápasov, z ktorých sú následne kalkulované ostatné informácie.
Nebudú použité abstraktné data ako sila hráčov alebo tímu ani ohodnotenia žiadnych hráčov ako ani data o počte rohových kopov, žltých kariet, es alebo nevynútených chýb. 
V tomto ohľade je najpodobnejšia práca od iránskych výskumníkov \citep{related:iran}, ale aj tam sú značné rozdiely v použití dat.

Žiadna z vyššie spomínaných prác nepoužíva ako jednu z metód vyhodnocovania sietí kurzy stávkových kancelárií.
V tejto práci nás hlavne zaujímajú zápasy, v ktorých ani jeden z tímov nie je favoritom z hľadiska kurzov stávkových kancelárií.
Vyhodnocovať teda budeme celkovú úspešnosť siete; zisk, ktorý by sme dosiahli pri stávkovaní na všetky zápasy a zisk, ktorý by sme dosiahli stávkovaním výhradne na zápasy, v ktorých nie je jasný favorit.


\include{chap01}
\chapter{Neurónové siete}

\iffalse
Formálne  je neurónová    sieť    určená    ako    orientovaný    graf    G=(V,E).    
Výrazy V={v1,v2,...,vN}  a  E={e1,e2,...eM}  označujú  neprázdnu  vrcholovú  množinu  resp.  hranovú množinu grafu G obsahujúceho N vrcholov (neurónov) a M hrán (spojov). Každý spoj $e \in E$ sa interpretuje ako usporiadaná dvojica dvoch neurónov z množiny V, e=(v,v’).  
Hovoríme, že  spoj  e  začína  v  neuróne  v  a  končí  v  neuróne  v’.  Množina  neurónov  V  je  rozložená  na disjunktné podmnožiny.

kde VI obsahuje NI vstupných neurónov, ktoré sú susedné len s vychádzajúcimi hranami, VH obsahuje NH skrytých (angl. hidden) neurónov, ktoré sú susedné súčasne  s  vychádzajúcimi ako aj s vchádzajúcimi hranami, a konečne VO obsahuje NO výstupných neurónov, ktoré súsusedné  len  s  vchádzajúcimi  hranami.  
V  našich  nasledujúcich  úvahách  budeme  vždy predpokladať,  že  množiny  VI  a  VO  sú  neprázdne,  t.j.  neurónová  sieť  obsahuje  vždy  aspoň jeden vstupný a jeden výstupný neurón.
Pre acyklické neurónové siete (ktoré neobsahujú orientované cykly neuróny môžu byť usporiadané do vrstiev VL  L  $LL=\cup \cup \cup \cup 12 3...t$, kde L1=VI je vstupná vrstva (obsahuje len vstupné neuróny), L2, L3,..., Lt–1 sú skryté vrstvya  Lt  je  výstupná  vrstva.  
Vrstva  Li  (pre  $1\leq i\leq t$)  je  určená  nasledujúcim  jednoduchým spôsobom $LVivdvi=\in   =+;aflq1$, kde vzdialenosť d(v) sa rovná dĺžke maximálnej cesty, ktorá spája daný neurón so vstupným neurónom, potom musí platiť d(v)=0, pre $v\in VI$. 
Neurónová sieť určená acyklickým grafom je  obvykle  volená  tak,  že  neuróny  z  dvoch  susedných  vrstiev  sú  poprepájané  všetkými možnými  spojmi.  
Žiaľ,  takýto  rozklad  množiny  neurónov  na  vrstvy  je Obrázok 5.2. 
Neurónová sieť je definovaná ako orientovaný súvislý graf. 
Diagram A obsahuje orientovaný graf  s  jedným  cyklom  a  teda  nemôže  byť  použitý  pre  definíciu  neurónovej  siete  s  dopredným  šírením.
Diagram  B  ilustruje  možnosť  rozkladu  vrcholov  (neurónov)  acyklického  orientovaného  grafu  na  vrstvy L1,..,L4.
kde L1=VI je vstupná vrstva (obsahuje len vstupné neuróny), L2, L3,..., Lt–1 sú skryté vrstvya  Lt  je  výstupná  vrstva.  Vrstva  Li  (pre  $1\leq i\leq t$)  je  určená  nasledujúcim  jednoduchýmspôsobom$LVivdvi=\in   =+$;aflq1                                  (5.9)kde vzdialenosťd(v) sa rovná dĺžke maximálnej cesty, ktorá spája daný neurón so vstupnýmneurónom, potom musí platiťd(v)=0, pre $v\in VI$. Neurónová sieť určená acyklickým grafomje  obvykle  volená  tak,  že  neuróny  z  dvoch  susedných  vrstiev  sú  poprepájané  všetkýmimožnými  spojmi  (pozri  obr.  5.3). \citep{rnn:spol}

\fi

Neurónová sieť je založená na orientovanom grafe (ako je možné vidieť na obrázku \ref{fig:nn}), je teda zložená z uzlov, ktoré sú spojené orientovanými hranami \citep{rnn:spol}.
Spojenie uzlu $i$ do uzlu $j$ slúži na propagáciu aktivácie $a_i$ z $i$ do $j$.
Každé takéto spojenie má priradenú váhu $w_{i,j}$, ktorá rozhoduje o sile a znamienku spojenia.
Každý uzol má naviac falošný vstup $a_0=1$ s priradenou váhou $w_{0,j}$.
Všetky uzly si potom vypočítajú váženú hodnotu vstupov, pre uzol $j$ je táto hodnota:
$$in_j=\sum^n_{i=0}w_{i,j}a_i$$
Potom sa na výsledok aplikuje aktivačná funkcia g, tým získame výstup z uzlu:
$$a_j=g(in_j)=g\left(\sum^n_{i=0}w_{i,j}a_i\right)$$

\begin{figure} \label{fig:nn}
\includegraphics[width=\textwidth]{../img/nn.png}
\caption{Na obrázku je ukážka jednej z neurónových sietí. Konkrétne to je sieť, ktorú použili v roku 2014 iránski výskumníci pri predikcii výsledkov najvyššej iránskej ligy \citep{related:iran}.}
\end{figure}

Aktivačná funkcia $g$ je typicky buď pevná hranica alebo logistická funkcia.
V prvom prípade sa uzly volajú perceptrony, v druhom prípade sa niekedy používa pojem sigmoid perceptron.
Obe tieto typy nelineárnych aktivačných funkcií zaručujú dôležitú vlastnosť neurónovej siete, a to, že celá sieť uzlov môže reprezentovať aj nelineárnu funkciu.

\begin{figure}
\includegraphics[width=\textwidth]{../img/nn_aima_neuron.png}
\caption{Takto vyzerá jeden uzol siete (neurón) \citep{aima}.}
\end{figure}

Takto teda vyzerá matematický model jedného uzlu (v tomto prípade zvaného neurón) v sieti.
Spájanie týchto neurónov vytvorí sieť.
Existujú dva rozdielne prístupy, akými sa dajú tieto neuróny spojiť do siete. 
Obe nás zaujímajú pre túto prácu, pretože obe použijeme v praxi a budeme ich porovnávať medzi sebou.

Neurónové siete sú obvykle zoradené do vstiev tak, že každý neurón dostane vstup len z neurónov z predošlej vrstvy.
Podľa počtu vrstiev sa siete delia na jednovrstvové a viacvrstvové, kde jednovrstvové siete spájajú vstpné neuróny priamo s výstupnými.
Viacvrstvové siete majú medzi vstupom do siete a výstupom z nej ešte jednu alebo viac vrstiev tzv. skrytých (hidden) neurónov (Obrázok \ref{img:single}) \citep{aima}.

\begin{figure} \label{img:single}
\includegraphics[width=\textwidth]{../img/nn_aima_single_multi.png}
\caption{Ukážka rozdielu medzi jednovrstvou sieťou (a) a viacvrstvovou (b). Obe majú 2 vstupné a 2 výstupné neuróny, viacvrstvová má ešte medzi nimi ďalšie vrstvy skrytých neurónov (v tomto prípade jednu vrstvu s 2 skrytými neurónmi) \citep{aima}.}
\end{figure}

\section{Dopredné neurónové siete}
Dopredná neurónová sieť (feed-forward neural network) má spojenia len v jednom smere, takže tvorí orientovaný acyklický graf.
Ak si graf topologicky usporiadame, tak každý uzol dostane vstup z niektorých z predchádzajúcich uzlov a predá výstup niektorým z nasledujúcich vrcholov.
Dopredná neurónová sieť teda predstavuje funkciu jej momentálneho vstupu, teda neuchováva žiaden stav, ak nepočítame váhy samotné \citep{aima}.

\section{Rekurentné neurónové siete}
Na druhej strane je rekurentná neurónová sieť (recurrent neural network).
Tento typ siete posúva svoj výstup naspäť do svojho vlastného vstupu. 
Z toho vyplýva, že aktivačné úrovne siete tvoria dynamický systém, ktorý môže dosiahnuť stabilný stav alebo oscilovať či sa dokonca správať chaoticky.

Výstup siete závisí na vstupe. 
Pri tomto type siete môže výstup závisieť aj na predchádzajúcich výstupoch, tranzitívne teda aj na predchádzajúcich vstupoch.
Z toho vyplýva, že si rekurentná neurónová sieť môže vypracovať krátkodobú pamäť \citep{aima}.
\chapter{Datasety}

Data, s ktorými budeme pracovať, sú výhradne len výsledky a konečné stavy jednotlivých zápasov. 

\section{Futbal}
Pre futbal data predstavujú pre každú ligu dataset všetkých zápasov odohraných len vrámci ligy za pár posledných sezón. 
Nebudeme používať žiadne data informujúce o hráčoch, ktorí sú v oficiálnej súpiske na zápas ani data priamo len o základnej zostave na daný zápas. 
Taktiež vzhľadom na to, že tímy v jednotlivých ligách hrajú zápasy aj mimo ligy, prinajmenšom zápasy v ligovom pohári, tak nebudú použité ani informácie o oddychu pred daným zápasom, teda koľko dní pred zápasom mali zúčastnené tímy voľno.

Dataset pre každú ligu je tabuľka, kde riadky predstavujú jednotlivé zápasy zoradené podľa dátumu, v ktorom bol zápas odohraný, zostupne.
Stĺpce sú v poradí:
\begin{enumerate}
  \item Jednoznačný názov domáceho tímu (nemusí byť celý názov, stačí skrátený, ale jednoznačný a, pokiaľ možno, v celom datasete konzistentný),
  \item Jednoznačný názov hosťujúceho tímu,
  \item Id zápasu,
  \item Ligové kolo, v ktorom sa zápas odohral (0, ak sa nevie),
  \item Id domáceho tímu,
  \item Id hosťujúceho tímu,
  \item Počet gólov strelených domácim tímom v zápase,
  \item Počet gólov strelených hosťujúcim tímom v zápase,
  \item Dátum zápasu,
  \item Sezóna,
  \item Kurz na výhru domácich,
  \item Kurz na remízu,
  \item Kurz na výhru hostí.
\end{enumerate}

Tento dataset potom predáme programu \textit{DataMaker.exe} (TODO!) písanom v jazyku C\#, ktorý pretransformuje tieto data na vstupné neuróny pre neurónovú sieť. Všetkých vstupných neurónov je 44, v poradí: 
\begin{multicols}{2}
\begin{enumerate}
 \item htW,
 \item htD,
 \item htL,
 \item htGFpG,	 
 \item htGApG,	 
 \item atW,	 
 \item atD,	 
 \item atL,	 
 \item atGFpG,	 
 \item atGApG,	 
 \item htHW,	 
 \item htHD,	
 \item htHL,	 
 \item htHGFpG,	 
 \item htHGApG,	 
 \item atAW,	 
 \item atAD,	 
 \item atAL,	 
 \item atAGFpG,	 
 \item atAGApG,	 
 \item hFW,	 
 \item hFD,	 
 \item hFL,	 
 \item hFGF,	 
 \item hFGA,	 
 \item aFW,	 
 \item aFD,	 
 \item aFL,	 
 \item aFGF,	 
 \item aFGA,	 
 \item MW,	 
 \item MD,	 
 \item ML,	 
 \item MGF,	 
 \item MGA,	 
 \item MhW,	 
 \item MhD,	 
 \item MhL,	 
 \item MhGF,	 
 \item MhGA,	 
 \item htLTS,	 
 \item atLTS,	 
 \item dFS,	 
 \item dFCS.
\end{enumerate}
\end{multicols}
Vysvetlivky: Prefixy: h[t] -- domáci tím, a[t] -- hosťujúci tím, F -- forma (posledných 5 zápasov), M -- posledných 5 vzájomných zápasov oboch daných tímov, Mh -- posledných 5 vzájomných zápasov oboch tímov hrané na ihrisku domáceho tímu.
Sufixy: W -- počet výher, D -- počet remíz, L -- počet prehier, GF[pG] -- počet strelených gólov prepočítaných na zápas, GA[pG] -- počet inkasovaných gólov prepočítaných na zápas, LTS -- dlhodobá sila tímu (priemerný počet bodov tímu v posledných sezónach v lige).
Ďalšie: dFS -- rozdiel v skóre formy oboch tímov, dFCS -- rozdiel v momentálnom skóre formy oboch tímov. 
Skóre formy oboch tímov je vypočítané ako počet bodov súpera v posledných 5 zápasov pre tím vynásobený počtom bodov získaných z daného zápasu. 
Momentálne skóre formy funguje podobne s výnimkou toho, že to je prepočítavané pred momentálnym zápasom, zatiaľ čo skóre sa počíta v momente ukončenia zápasu.

Skóre je pokus čo najlepšie ohodnotiť formu tímu jedným údajom. 
V sekcii (Vylepšovanie siete) (TODO!) sa budeme snažiť znížiť počet vstupných neurónov a ponechať len tie, ktoré sú dôležité. Čím väčší počet vstupných neurónov, tým väčšia je šanca, že sieť sa pokúsi medzi datami nájsť nejakú súvislosť, ktorá tam nie je, čo môže pri testovacích datach vyústiť v nesprávne výsledky (pretrénovanie dat) (TODO! cit).\\

Posledné 3 stĺpce tohto súboru predstavujú kurzy na dané výsledky. Tieto ale nie sú pri trénovaní siete využívané.

Program tiež vytvorí ďalší súbor, ktorý obsahuje testovacie data, teda data, ktoré sa nevyužívajú pri trénovaní siete, ale len pri vyhodnocovaní výsledkov. 
Tieto data sú v rovnakom poradí a musia obsahovať kurzy na dané výsledky a aj výsledok zápasu vo forme troch stĺpcov v poradí domáci, remíza, hostia, kde výsledok, ktorý nastal je ohodnotený 1, zvyšné sú 0. 
Je to potrebné pre vyhodnocovanie, pretože neurónová sieť bude mať 3 výstupné neuróny v rovnakom poradí a predikciu ohodnotí na 1.

Data v jednom súbore predstavujú pár posledných sezón a prvú polovicu sezóny 2018/2019, ktorá predstavuje všetky odohrané zápasy od začiatku sezóny až po odohratie posledného zápasu pred začiatkom kola, ktoré je numericky už v druhej polovici sezóny. 
Napríklad najvyššia anglická futbalová liga Premier League má 38 kôl každú sezónu, do úvahy sa bude brať posledných pár sezón pred sezónou 2018/2019 a všetky zápasy odohrané pred prvým zápasom 20. kola sezóny 2018/2019 (s výnimkou predohrávok, teda zápasov, ktoré boli preložené na dátum pred dátumom, v ktorom daný zápas figuroval v predsezónnom rozpise zápasov). 
Túto hranicu pre každú predikovanú ligu uvediem ručne do zdrojového kódu programu DataMaker.exe, pretože neviem o nejakom reálnom funkčnom algoritme, ktorý by to vedel s absolútnou istotou určiť a predstavuje to len jednu sezónu pre 5 líg.

Tento súbor je potom predaný programu v0.py (TODO!), ktorý data pripraví, vytvorí neurónovú sieť s danými parametrami (bližšie o presných parametroch v kapitole Príprava siete) a naučí ju dané data, ktoré nakoniec vyhodnotí podľa rôznych kritérií ako dôvera v daný tip alebo kurzovo vyrovnané zápasy, teda zápasy, kde na výhru domácich a vyhru hostí je dostatočne podobný kurz.

\section{Tenis}
Pre tenis budeme používať data pre najlepších 100 hráčov na začiačtku každého roka v rebríčku ATP. Data v tomto prípade predstavujú zápasy z turnajov typu ATP 250, ATP 500, ATP Masters 1000, Grandslam, Finals, Nextgen Finals a ... .

Dataset je tabuľka, každý zápas predstavuje jeden riadok tabuľky, zápasy sú zoradené do turnajov od najskôr odohraných turnajov po tie najbližšie súčasnosti (ak sa obe turnaje začali a končili hrať v rovnaký deň, tak sú v ľubovoľnom poradí, nie je možné, aby poradie zmenilo nejaké data, pretože nie je možné hrať na dvoch turnajoch takéhoto typu zároveň). 
Zápasy v turnajoch sú zoradené od finále po prvé kolo, teda intuitívne opačne. 
V tomto prípade na poradí nezáleží, dôležité je, že je v tom systém. Program na spracovanie dat (ATPDataMaker.exe) si tie poradie dat upraví tak, aby mu vyhovovali. 
Stĺpce tabuľky sú v poradí:
\begin{enumerate}
  \item Názov turnaja,
  \item Počet bodov, ktoré víťaz obrdží za výhru v turnaji (ak to je neznáme, tak je tam nápis N/A)
  \item Rok, v ktorom sa turnaj odohral,
  \item Povrch kurtov na turnaji (tvrdý, antukový alebo trávnatý povrch),
  \item Meno víťaza zápasu,
  \item Meno hráča, ktorý zápas prehral,
  \item Kolo turnaja, v ktorom sa zápas odohral od najdôležitejšieho (1 značí finále, 2 semifinále, apod.),
  \item ID zápasu,
  \item ID víťaza,
  \item ID porazeného hráča,
  \item Počet setov, ktoré v zápase získal víťaz,
  \item Počet setov, ktoré v zápase získal porazený hráč,
  \item Počet hier, ktoré v zápase získal víťaz v jednotlivých setoch oddelené znakom |,
  \item Počet hier, ktoré v zápase získal porazený hráč v jednotlivých setoch oddelené znakom |.
\end{enumerate}

ID hráčov sa nachádzajú v ďalšom súbore (\textit{atpranking.csv}), ktorý sa predáva aplikácii na tvorbu vstupných neurónov do neurónovej siete.
Tento súbor obsahuje ID jednotlivých hráčov, ich mená a ich poradie v koncoročných rebríčkoch hodnotenia ATP za roky 1999--2018.
Poradie berieme len ak sa hráč umiestnil na miestach 1--100.

Zápasy obsiahnuté v súbore \textit{atpresults.csv} sú len zápasy, v ktorých aspoň jeden hráč bol na konci aspoň raz v daných rokoch na miestach 1--100 v hodnotení ATP.
Predikovať sa budú len zápasy medzi takýmito hráčmi, ale kvôli rôznym výpočtom je potrebné mať všetky data o takýchto hráčoch z turnajov, ktoré sú obsiahnuté v súbore.

Tieto datasety sa potom predajú súboru \textit{ATPDataMaker.exe}, ktorý ich pretransformuje na data pre vstupné neuróny neurónových sietí. Všetkých vstupných neurónov je 37, súbor ku každému vstupnému neurónu vydá aj očakávaný výstup (1?, 2?) a pre predikovanú časť dodá aj kurzy stávkových kancelárií na daný výsledok (1B, 2B).
Výstupné súbory majú teda stĺpce v poradí:
\begin{multicols}{2}
\begin{enumerate}
 \item 1W
 \item 1L
 \item 1GDpS
 \item 2W
 \item 2L
 \item 2GDpS
 \item 1FW
 \item 1FL
 \item 1FGDpS
 \item 2FW
 \item 2FL
 \item 2FGDpS
 \item 1SW
 \item 1SL
 \item 1SGDpS
 \item 2SW
 \item 2SL
 \item 2SGDpS
 \item 1SFW
 \item 1SFL
 \item 1SFGDpS
 \item 2SFW
 \item 2SFL
 \item 2SFGDpS
 \item 1MW
 \item 1ML
 \item 1MGDpS
 \item 1MSW
 \item 1MSL
 \item 1MSGDpS
 \item 1R
 \item 2R
 \item H
 \item C
 \item G
 \item dSc
 \item dSSc
 \item 1?
 \item 2?
\end{enumerate}
\end{multicols}
Vysvetlivky: Prefixy: {1,2} -- hráči, M -- vzájomné zápasy (z pohľadu hráča 1), S -- povrch (zápasy hráča na tomto povrchu), F -- forma (posledných 10 zápasov).
Suffixy: W -- počet výhier*, L -- počet prehier*, GDpS -- priemerný rozdiel v počte získaných hier v sete v prospech daného hráča, R -- poradie v rebríčku, ? -- výsledok (ak hráč vyhral - 1, inak 0)
Zvyšné: H -- tvrdý povrch, C -- antuka, G -- trávnatý povrch, dSc - rozdiel v skóre** medzi hráčmi (z pohľadu hráča 1), dSSc - rozdiel v povrchovom skóre** (z pohľadu hráča 1)\\
* - v danej sezóne, s výnimkou, ak predchádza prefix SF - povrch si uchováva formu hráča aj z minulej sezóny, ak bola braná do úvahy
** - skóre je pokus ohodnotiť silu víťazstva, berie do úvahy formu, teda posledných 10 zápasov a počíta sa ako $(150 - rank)\cdot point$, kde rank je poradie súpera v poslednom koncoročnom rebríčku ATP a point je 1, ak hráč vyhral, 0, ak vyhral súper. 
Ak súper nebol v Top 100 rebríčka ATP na konci predchádzajúceho roka, tak za jeho rank je dosadené číslo 130.
To je len preto, lebo teoreticky má dané víťazstvo hodnotu, musí byť teda nejak ohodnotené lepšie ako ľubovoľná prehra, ktorá je ohodnotená hodnotou 0.

Skóre je pokus výraznejšie ohodnotiť formu hráča ako len počtom výhier a prehier. Pri pokusoch a vylaďovaní siete budeme v sekcii (Vylepšovanie siete) (TODO!) selektovať dané vstupné neuróny podľa rôznych kritérií a vyskúšame tiež aj ako sa bude sieť správať, ak nahradíme všetky stĺpce obsahujúce data o forme rozdielom v skóre.
Teoreticky tým ušetríme 6 vstupných neurónov, príliš veľa vstupných neurónov môže viesť k rýchlejšiemu pretrénovaniu siete (TODO! cit), čomu sa budeme snažiť zabrániť selektovaním len tých dôležitých neurónov.
\chapter{Stavba siete} \label{stavba}

Všetky siete boli napísané v programovacom jazyku \textit{Python} s použitím knižníc \textit{numpy} a \textit{tensorflow}.
Na vylepšovanie siete sme, ako je napísané v kapitole \ref{docu}, použili údaje, ktoré sa napokon budú pri vyhodnocovaní nachádzať medzi trénovacími dátami. 

Konkrétne, pre futbal dáta predstavovali 7 celých sezón a prvú polovicu ďalšej sezóny (ako popísané v \ref{foot}), vyhodnocovacie dáta predstavujú druhú polovicu tejto sezóny. 
Takže sme trénovacie dáta rozdelili na tri časti, 6 celých sezón a polovicu ďalšej (trénovacie dáta), druhú polovicu siedmej sezóny (testovacie dáta) a zvyšnú prvú polovicu ôsmej sezóny (nepoužité dáta).

V prípade tenisu prišli dáta už priamo z transformačnej časti v troch súboroch, trénovacie, testovacie a vyhodnocovacie dáta.

Každá sieť mala svoje nedostatky v celkovej úspešnosti, ale doposiaľ neexistuje efektívne nastavenie neurónových sietí pre každú situáciu \citep{gitgud}, takže každá sieť sa musela vylepšovať osobitne a manuálne vzhľadom na rozdiely v prístupoch.

\section{Selekcia príznakov}
Kliatba dimenzionality nám hovorí, že čím viac vstupných príznakov zadáme neurónovej sieti, tým sú dáta redšie a teda je ich potreba získať viac, aby sa sieť správne učila. Viac dát získať nevieme, takže sa pokúsime znížiť počet dimenzií a pozrieť sa na to, ako sa to prejaví na trénovacích dátach.
Na začiatok spravíme korelačný test všetkých príznakov testovacích a trénovacích dát (obrázky \ref{corr} a \ref{corr_atp}).
Teória hovorí, že by nám mohla napovedať, aké hodnoty sú dôležité.
Obrázok \ref{corr} týkajúci sa futbalu hovorí, že najvyššiu koreláciu s výsledkom zápasu majú príznaky 41 a 42 určujúce dlhodobú silu tímu. Skóre dosahuje tiež celkom vysokú koreláciu (okolo 0,2) s finálnym výsledkom.
\noindent
\begin{figure}  [h!]
\includegraphics[scale=0.9]{../img/correng.png}
\caption{Korelačná tabuľka všetkých príznakov pre anglickú Premier League, žltá predstavuje kladnú koreláciu, modrá zápornú. Nás hlavne zaujímajú posledné 3 riadky určujúce koreláciu príznaku s výsledkom zápasu (príznaky sú v poradí ako v Prílohe \ref{in:foot}).}
\label{corr} 
\end{figure}

V prípade tenisu obrázok \ref{corr_atp} naznačuje najvyššiu koreláciu medzi výsledkom a oboma druhmi skóre, veľkú rolu zohráva postavenie v rebríčku ATP a vzájomné zápasy (konkrétne priemerný rozdiel v počte vyhraných gemov za set).

\begin{figure} [h!]
\includegraphics[scale=0.9]{../img/corratp.png}
\caption{Korelačná tabuľka všetkých príznakov pre tenisové zápasy, žltá predstavuje kladnú koreláciu, modrá zápornú. Nás hlavne zaujímajú posledné 2 riadky určujúce koreláciu príznaku s výsledkom zápasu (príznaky sú v poradí ako v Prílohe \ref{in:foot}).}
\label{corr_atp}
\end{figure}

Tieto tabuľky boli pre nás viac-menej informačné.
Korelovanosť jedného príznaku s výsledkom nám nemusí hovoriť nič.
Stačí sa pozrieť na funkciu XOR (tabuľka \ref{xor}). Ak by sme na hodnoty v tejto tabuľke aplikovali koreláciu, tak by sme sa dozvedeli, že x, y a ani hodnota x $XOR$ y nemajú po dvoch žiadnu koreláciu (ak vieme, akú hodnotu má x, nijak to pre nás nemení pravdepodobnosť, že x XOR y bude napríklad 0). Až kombinácia hodnôt x a y priamo určuje hodnoty funkcie XOR.
Selekcia príznakov, s ktorými dosahovala sieť najlepšie trénovacie výsledky a ktoré budú použité na získanie výsledkov v kapitole \ref{res}, bol uskutočnený spôsobom pokus-omyl, keďže nič lepšie nevieme, ako už bolo spomínané na začiatku kapitoly.

\section{Proces tréningu} \label{train}
Proces tréningu siete začínal základným nastavením siete, ktoré je presnejšie popísané v nasledujúcich kapitolách, pretože pre každú sieť bolo unikátne.
Po trénovaní siete v základnom nastavení nasledovalo spustenie siete na testovacích dátach.
Nasledoval pokus nastaviť parametre (tie budú tiež popísané v nasledujúcich sekciách, líšili sa pre jednotlivé druhy neurónových sietí) jednotlivých sietí pre každý šport a ligu tak, aby dosahovali čo najvyššie sledované hodnoty.
Tieto hodnoty predstavovali referenčnú hodnotu, ktorú sme chceli selektovaním jednotlivých parametrov vylepšiť.
Následne sme vybrali jednu z podmnožín príznakov (napríklad odstránili údaje o forme) a opakovali proces vylaďovania parametrov.
Na konci pre každú sieť ostala jedna podmnožina príznakov a nastavenie parametrov siete, ktoré v kombinácii s ňou vydávalo najlepšie výsledky v sledovaných oblastiach.

Sledované oblasti zo začiatku predstavovali trénovaciu a testovaciu úspešnosť, úspešnosť v zápasoch bez favorita a eventuálny zisk, ktorý by sme dosiahli, ak by sme v týchto zápasoch uzatvárali stávky podľa nápovedy danej siete. 
Všetky modely ale ukazovali veľmi podobnú úspešnosť v tipovaní víťazov zápasov bez favorita (napríklad pri vytváraní siete pre tipovanie anglickej futbalovej Premier League vydávali všetky siete trénovaciu úspešnosť 36--39\%).
Eventuálny zisk osciloval bez ohľadu na túto úspešnosť.
Neobjavil som žiadnu koreláciu medzi nastavenými parametrami, selektovanými príznakmi a týmto ziskom, takže predpokladám, že na hodnote tohto údaja pri trénovaní až tak nezáleží, pretože vo vyhodnocovacom procese môže dosiahnuť úplne iné výsledky.
Nakoniec teda som sledované oblasti zúžil na trénovaciu a testovaciu úspešnosť, kde som sa pokúšal maximalizovať testovaciu úspešnosť a trénovacia slúžila hlavne ako referenčný údaj pre pretrénovanie siete (vysvetlené v sekcii \ref{learn}), aby som vedel, kedy už netreba pridávať ďalšie trénovacie iterácie vybranej podmnožiny trénovacích dát (\textit{epoch}), pretože úspešnosť sa už ďalej bude len znižovať.

Je potrebné dodať, že celý trénovací proces prebiehal v iteráciách.
Každý model siete bol trénovaný a vyhodnotený 40-krát (s rôznym náhodným nastavením váh spojení) a výsledky v sledovaných oblastiach boli aritmetickým priemerom cez týchto 40 modelov.
V prípade tenisu som do tohto aritmetického priemeru nepočítal dáta, ktoré sa od začiatku \uv{zasekli}, čo bolo jednoznačne viditeľné už z trénovacej úspešnosti a trénovacej chyby. 
Zaseknutie vyzeralo asi tak, že sieť zistila, na ktorej strane vyhráva viac hráčov (napríklad hráč 1 vyhráva v 50,2\% prípadov) a ani ďalšie učenie nepomohlo zmeniť názor siete. 
Fakt, že to bolo viditeľné už z trénovacích dát (obvykle pri tenise obe druhy neurónových sietí dosiahli viac ako 70\% trénovaciu úspešnosť), tak sme mohli tieto pokusy odstrániť a vylepšiť tým úspešnosť siete bez toho, aby sme nechali sieť vyhodnocovať testovacie dáta.
Ak by sme chceli prakticky použiť neurónové siete na predikciu výsledkov v tenise, tak by sme si tohto faktu všimli ešte predtým, ako by sme stavili na prvý zápas.

\section{Dopredné neurónové siete} \label{ffnn:train}
V prípade futbalu bola na začiatku sieť skonštruovaná so všetkými 44 príznakmi, ktoré sme dostali z transformačnej časti práce (príloha \ref{in:foot}).
Sieť teda obsahovala vstup o veľkosti 44 príznakov, vrstva regularizačnej techniky zvanej dropout (s jeho nastavením na 50\%), 2 skryté vrstvy neurónov (s 25, resp. 15 neurónmi) a troj-neurónový výstup typu softmax, ktorý vyberie najpravdepodobnejšiu možnosť, nastaví daný výstup neurónu na 1 a zvyšné nastaví na 0.

V prípade tenisu bola sieť skonštruovaná so všetkými 37 príznakmi (ich popis je príloha \ref{in:ten}) a taktiež obsahovala vrstvu \textit{dropout} regularizácie, 2 skryté vrstvy neurónov s rovnakým počtom neurónov na nich ako v prípade futbalu. 
Výstup predstavoval dva neuróny, na ktoré bol opäť použitý softmax, ktorý vyberie najvyššiu hodnotu.

Parametre siete, ktoré boli počas tréningu prenastavované a vylaďované sú:
\begin{enumerate}
  \item Spôsob, ktorým sa vylaďujú jednotlivé váhy siete (\textit{optimizer}),
  \item Počet iterácií trénovacieho procesu (\textit{počet trénovacích epoch}),
  \item Veľkosť jedného kroku trénovania (\textit{batch size}),
  \item Veľkosť modifikácie pri učení (\textit{learning rate}),
  \item Hodnota náhodne vypustených spojení po prvej vrstve (\textit{dropout}),
  \item Počet neurónov v prvej skrytej vrstve,
  \item Počet neurónov v druhej skrytej vrstve.
\end{enumerate}

Nastavenia jednotlivých parametrov, ktoré prinášali najlepšie výsledky pre každú sieť, sa nachádzajú v tabuľke \ref{rnn_train_res}.

\begin{table}[h!]
\begin{center}
\begin{tabular}{ p{7em}|c|c|c|c|c|c|c|c|c| } 
 Predpovedaný šport (liga) & \multicolumn{7}{|c|}{Parametre siete} & \multicolumn{2}{|p{5em}|}{Trénovacie výsledky}  \\ 
 \hline
  & O & E & B & LR & D & $H_1$ & $H_2$ & TrA\% & TeA\% \\
 \hline \hline
 Futbal (ENG) & GD & 100 & 64 & 0,005 & 0,5 & 15 & 10 & 52,9 & 50,9  \\ 
 Futbal (GER) & GD & 75 & 16 & 0,005 & 0,5 & 15 & 10 & 50,01 & 50,68  \\ 
 Futbal (SPA) & GD & 100 & 64 & 0,005 & 0,5 & 15 & 10 & 55,36 & 52,78  \\ 
 Tenis & A & 30 & 256 & 0,01 & 0,5 & 15 & 10 & 73,98 & 65,8  \\ 
 \hline
\end{tabular}
\caption{Tabuľka nastavenia parametrov doprednej neurónovej siete, pri ktorých dávala sieť najlepšie trénovacie výsledky a aj hodnoty, ktoré dosiahli v sledovaných oblastiach. Skratky v hlavičke tabuľky pod parametrami siete sú skratky sledovaných parametrov zo zoznamu (v tom istom poradí, v akom sú v zozname uvedené), trénovacie výsledky sú trénovacia úspešnosť a testovacia úspešnosť v tomto poradí. V stĺpci O (\textit{optimizer}) skratka GD znamená metódu \textit{Gradient Descent} a skratka A metódu \textit{Adam}.}
\label{ff_train_res}
\end{center}
\end{table}

Už z týchto trénovacích výsledkov môžeme vidieť, že model, ktorý dosahoval najlepšie výsledky sa dosť líši pri predikcii rôznych športoch. 
Celkovo pre doprednú neurónovú sieť bolo vytvorených viac ako 200 rôznych modelov, aj preto je zvláštnosťou, že \textit{Gradient Descent} algoritmus vydával lepšie výsledky pri predikcii futbalu ako jeho vylepšená verzia \textit{Adam}.

Rôzne modely sa líšili aj selekciami rôznych príznakov.
V prípade futbalu sa dokonca vybrané príznaky líšili aj na predikovanej lige.
Pre anglickú (ENG) a španielsku ligu (SPA) dosahovala najlepšie výsledky sieť, ktorá používala 21 z 44 príznakov. 
Použité boli všetky príznaky okrem príznakov určujúcich formu, priemerný počet gólov (strelených aj inkasovaných) na zápas a momentálne skóre (konkrétne boli použité príznaky 1, 2, 3, 6, 7, 8, 11, 12, 13, 16, 17, 18, 31, 32, 33, 36, 37, 38, 41, 42 a 43 z prílohy \ref{in:foot}). Tieto vedomosti nám poukazujú na fakt, že som bol úspešný pri nahradení formy jedným z druhov skóre a ušetril som tým 9 dimenzií.
Rozdiel v úspešnosti, ak bolo použité skóre a ak nebolo použité skóre bol minimálny, čo poukazuje na fakt, že skóre bolo užitočné, ale nie perfektné a určite by sa dalo nejak vylepšiť.

Pre nemeckú ligu (GER) uvedená sieť používala 19 príznakov. Použité boli všetky príznaky okrem príznakov určujúcich počet remíz vo všetkých kategóriách, priemerný počet gólov na zápas a rozdiel v momentálnom skóre (konkrétne použité príznaky boli príznaky s číslami 1, 3, 6, 8, 11, 13, 16, 18, 21, 23, 26, 28, 31, 33, 36, 38, 41, 42 a 43 z prílohy \ref{in:foot}).

Pre tenis dosahovali najlepšie výsledky siete, ktoré používali 27 príznakov.
Príznaky, ktoré neboli potrebné v procese učenia sú povrch, na ktorom sa zápas odohrá, údaje o forme a skóre povrchu (použité teda boli príznaky s číslom 1, 2, 3, 4, 5, 6, 13, 14, 15, 16, 17, 18, 19, 20, 21, 22, 23, 24, 25, 26, 27, 28, 29, 30, 31, 32 a 36 z prílohy \ref{in:ten}).
Opäť vidíme, že sa nám podarilo nahradiť údaje o forme umelo (ale automaticky) vytvoreným skóre, čo mierne zvýšilo úspešnosť siete. Pri forme na povrchu sa to ale nepodarilo a lepšie výsledky dosahovala sieť bez použitia skóre povrchu a s použitím tejto skupiny príznakov.

\section{Rekurentné neurónové siete} \label{rnn:train}
Tréning rekurentnej neurónovej siete (\textit{RNN}) je všeobecne časovo náročnejší ako trénovanie doprednej neurónovej siete, nakoľko sa do procesu tréningu zapojí aj stav siete. Sieť sa musí naučiť, ktoré údaje sú dôležité na uchovávanie v stave siete.
Z tohto dôvodu som pri tréningu RNN vychádzal aj z údajov získaných v predošlej sekcii, pričom som ale celý proces tréningu (tak, ako je popísaný v sekcii \ref{train}) zachoval.
Základné nastavenie siete používalo všetky príznaky z príloh. 
Pre futbal ich bolo 44, pre tenis 37 (Prílohy \ref{in:foot} a \ref{in:ten} v poradí).
Tieto príznaky boli napojené do vrstvy 25 LSTM neurónov, potom nasledoval \textit{dropout} a výstupná vrstva.
Pár vyskúšaných nastavení siete obsahovalo pred výstupnou vrstvou ešte skrytú vrstvu neurónov.

Parametre siete, ktoré boli počas tréningu prenastavované a vylaďované sú:
\begin{enumerate}
  \item Spôsob, ktorým sa vylaďujú jednotlivé váhy siete (\textit{optimizer}),
  \item Počet iterácií trénovacieho procesu (\textit{počet trénovacích epoch}),
  \item Veľkosť jedného kroku trénovania (\textit{batch size}),
  \item Veľkosť modifikácie pri učení (\textit{learning rate}),
  \item Hodnota náhodne vypustených spojení po prvej vrstve (\textit{dropout}),
  \item Počet LSTM neurónov,
  \item Počet zápasov, pre ktoré si pri trénovaní sieť pamätá údaje (\textit{LSTM Timestamp}),
  \item Počet neurónov v skrytej vrstve (ak 0, tak model siete neobsahuje túto vrstvu).
\end{enumerate}

Nastavenia jednotlivých parametrov, ktoré prinášali najlepšie výsledky pre každú sieť, sa nachádzajú v tabuľke \ref{rnn_train_res}.

\begin{table}[b!]
\begin{center}
\begin{tabular}{ p{7em}|c|c|c|c|c|c|c|c|c|c| } 
 Predpovedaný šport (liga) & \multicolumn{8}{|c|}{Parametre siete} & \multicolumn{2}{|p{5em}|}{Trénovacie výsledky}  \\ 
 \hline
  & O & E & B & LR & D & N & T & H & TrA\% & TeA\% \\
 \hline \hline
 Futbal (ENG) & A & 70 & 64 & 0,005 & 0,5 & 25 & 30 & 10 & 52,44 & 51,95  \\ 
 Futbal (GER) & A & 100 & 64 & 0,005 & 0,5 & 25 & 30 & 10 & 49,66 & 51,21  \\ 
 Futbal (SPA) & A & 100 & 64 & 0,005 & 0,5 & 25 & 30 & 10 & 54,6 & 53,24  \\ 
 Tenis & A & 70 & 256 & 0,005 & 0,5 & 25 & 31 & 0 & 74,31 & 65,96  \\ 
 \hline
\end{tabular}
\caption{Tabuľka nastavenia parametrov RNN, pri ktorých dávala sieť najlepšie trénovacie výsledky a hodnoty, ktoré dosiahli v sledovaných oblastiach. Skratky v hlavičke tabuľky pod parametrami siete sú skratky sledovaných parametrov zo zoznamu (v tom istom poradí, v akom sú v zozname uvedené), trénovacie výsledky sú trénovacia úspešnosť a testovacia úspešnosť v tomto poradí. V stĺpci O (\textit{optimizer}) skratka A znamená metódu \textit{Adam}.}
\label{rnn_train_res}
\end{center}
\end{table}

Opäť môžeme z trénovacích výsledkov vidieť rozdiel medzi jednotlivými špor\-tmi.
Tentokrát ale v nastaveniach parametrov siete nie je žiadny rozdiel, ale je rozdiel pri vybraných príznakoch, s ktorými dosahovali siete najlepšie výsledky.
použitých bolo okolo 60 rôznych modelov a vedomosti z predchádzajúcej sekcie.
Pri všetkých druhoch rekurentných neurónových sietí dosahoval najlepšie výsledky optimalizačný algoritmus \textit{Adam}.

Jednotlivé vybrané príznaky pre anglickú aj nemeckú ligu boli rovnaké ako vybrané príznaky pre nemeckú ligu v prípade doprednej neurónovej siete. 
Použitých bolo 19 príznakov (konkrétne príznaky 1, 3, 6, 8, 11, 13, 16, 18, 21, 23, 26, 28, 31, 33, 36, 38, 41, 42 a 43 z prílohy \ref{in:foot}). Vyradené príznaky predstavovali údaje o počte remíz, priemernom počte gólov na zápas a rozdiele v momentálnom skóre.

V prípade španielskej ligy boli vybrané príznaky rovnaké ako pre dopredné neurónové siete.
Použitých bolo 21 príznakov (konkrétne 1, 2, 3, 6, 7, 8, 11, 12, 13, 16, 17, 18, 31, 32, 33, 36, 37, 38, 41, 42 a 43 z prílohy \ref{in:foot}). Vyradené príznaky boli údaje o forme, priemernom počte gólov na zápas a rozdiele v momentálnom skóre.

V prípade tenisu sa vybrané príznaky líšili od doprednej neurónovej siete. Z 37 príznakov bolo vybraných 32, vyradené príznaky boli povrch, na ktorom sa zápas odohral (príznaky číslo 33, 34 a 35 z prílohy \ref{in:ten}) a obe druhy skóre (príznaky číslo 36 a 37). V tomto prípade zlyhal koncept skóre a použitie žiadneho skóre na úkor skupiny príznakov, ktoré malo nahrádzať nepomohlo, dokonca sieť vydávala lepšie výsledky, ak skóre nevidela.


\chapter{Dokumentácia} \label{docu}

Z programátorského hľadiska je práca rozdelená na tri časti. 
Prvú časť predstavuje získavanie výsledkov a kurzov jednotlivých zápasov. 
Druhú časť práce predstavuje transformácia dat na údaje priamo použiteľné pri konštrukcii neurónových sietí.
Poslednú časť tvorí stavba daného typu neurónovej siete pre daný šport.

Transformačná časť v prípade tenisu rozdelí data na 3 časti, trénovacie data, testovacie data (pre optimalizovanie siete) a vyhodnocovacie data.
V prípade futbalu data rozdelí na 2 časti, trénovacie a vyhodnocovacie data.
Pre optimalizáciu som použil program podobný programu v poslednej časti, ktorý si trénovacie data rozdelíl podľa potreby (konkrétne na prvých 6 celých sezón a prvú polovicu ďalšej a druhú časť tvorí druhá polovicu tejto sezóny, až následne prebiehalo učenie siete).
Je teda zaručené, že žiadna sieť neuvidí vyhodnocovacie data vopred pred finálnym vyhodnocovaním.
Vyhodnocovacie data sú použité až pre získavanie výsledkov použitých v tejto práci, a teda použili sa až v poslednej fáze.


Údaje, ktoré sa objavia vo výstupe sú celková úspešnosť a celkový zisk, úspešnosť a zisk siete pri vyrovnaných zápasoch (a vyrovnaný zápas považujem zápas, kde kurzy na výhru jedného alebo druhého tímu sa líšia najviac o 1) a úspešnosť a zisk siete pri výhradnom tipovaní zápasov, na ktoré máme istú dôveru (od hodnoty, ktorá je počítaná ako rozdiel dvoch najvyšších čísel, ktoré sieť vydá na výstup, jednoducho povedané, rozdiel najpravdepodobnejšej a druhej najpravdepodobnejšej možnosti výsledku zápasu z hľadiska siete).
Táto hodnota dôvery bola tiež vyoptimalizovaná pre každú sieť/program osobitne.


\section{Futbal}

Údaje o týchto zápasoch sa dajú stiahnuť jednoducho, spustením programu oddscaper.py a zadaním skratky danej ligy pre futbal ako parameter.
Skratky sú:
\begin{enumerate}
\item ENG - najvyššia anglická liga (Premier League)
\item GER - najvyššia nemecká liga (Bundesliga)
\item SPA - najvyššia španielska liga (La Liga)
\end{enumerate}
Program stiahne všetky výsledky a kurzy pre všetky zápasy všetkých kompletných sezón tej-ktorej ligy zo stránky www.oddsportal.com.

Tento dataset potom predáme programu \textit{DataMaker.exe} (TODO!) písanom v jazyku C\#, ktorý pretransformuje tieto data na vstupné neuróny pre neurónovú sieť. Všetkých vstupných neurónov je 44, ich význam a poradie je popísané v sekcii Prílohy (\ref{in:foot}). Program taktiež vydá ako posledné tri stĺpce aj výsledok zápasu ako kategorické hodnoty v poradí domáci, remíza, hostia, kde výsledok, ktorý nastal je ohodnotený 1, zvyšné sú 0. (tiež popísané v Prílohe \ref{in:foot}).

Program tiež vytvorí ďalší súbor, ktorý obsahuje testovacie data, teda data, ktoré sa nevyužívajú pri trénovaní siete, ale len pri vyhodnocovaní výsledkov. 
Tieto data sú v rovnakom poradí a musia obsahovať kurzy na dané výsledky a aj výsledok zápasu vo forme troch stĺpcov 
Je to potrebné pre vyhodnocovanie, pretože neurónová sieť bude mať 3 výstupné neuróny v rovnakom poradí a predikovaný výsledok ohodnotí na 1.

Výstup tohto programu predáme programu \textit{DataMaker.exe} (teda ako prvý parameter programu DataMaker.exe je potrebné predať cestu k súboru, ktorý je výstupom súboru oddscraper.py, tento súbor sa volá rovnako ako skratka danej ligy s príponou \textit{.csv}), ktorý je písaný v jazyku C\# a pretransformuje tieto data na vstupné neuróny pre neurónovú sieť. 
Všetkých vstupných neurónov je 44. 
Presné poradie aj popis sa dá nájsť v sekcii Prílohy (Príloha \ref{in:foot}).
Tieto údaje boli vybrané špecificky aj s pomocou súvisiacich prác ako údaje, ktoré popisujú stav oboch tímov, ktoré hrajú proti sebe zápas. 
Bonus predstavujú vstupy označené ako skóre, tieto boli vytvorené mnou ako pokus o jednoduchý a presnejší popis formy pomocou jedného údaju namiesto 10.
Ak bude mať teda jeden z týchto neurónov (alebo obe spoločne) úspech, tak bude možné skrátiť počet vstupných neurónov o 10.
Pre upresnenie, výstupom súboru sú opäť tabuľky formátu csv, názov je zložený zo skratky pre názov danej ligy a slova \textit{input} pre trénovacie data, pre testovacie je to skratka danej ligy a slovo \textit{resinput}.

Cestu na dané súbory potom ako prvé dva parametre (v poradí, v akom sú uvedené v úvode kapitoly) predáme programu \textit{ffnnfootball.py} alebo \textit{rnnfootball.py} podľa toho, či chceme, aby dané údaje vyhodnocovala dopredná rekurentná neurónová sieť.
Výsledky vypíše na štandardný výstup a uloží ich aj do logu, ktorého názov pozostáva z typu siete, názvu ligy a časovej známky vo formáte \textit{txt}.

\section{Tenis}

Údaje o tenisových zápasoch sú predpripravené v súbore atpresults.csv.
Súbor \textit{atpranking.csv} je tiež predpripravený a obsahuje ID jednotlivých hráčov, ich mená a ich poradie v koncoročných rebríčkoch hodnotenia ATP za roky 1999--2018.
Poradie berieme len ak sa hráč umiestnil na miestach 1--100.
Predzápasové kurzy môžu byť prázdne (vyplnené kurzom 0.0), ale len, ak nás na daný zápas nezaujímajú kurzy (zaujímajú nás len pre posledné dva roky, prvý je na testovanie a druhý na vyhodnocovanie). 
Zápasy obsiahnuté v súbore \textit{atpresults.csv} sú len zápasy, v ktorých aspoň jeden hráč bol aspoň raz v daných rokoch na miestach 1--100 v koncoročnom hodnotení ATP.
Predikovať sa budú len zápasy medzi takýmito hráčmi, ale kvôli rôznym výpočtom je potrebné mať všetky data o takýchto hráčoch z turnajov, ktoré sú obsiahnuté v súbore.

Tieto datasety sa potom predajú súboru \textit{ATPDataMaker.exe}, ktorý ich pretransformuje na data pre vstupné neuróny neurónových sietí. Všetkých vstupných neurónov je 37 (ich presné poradie a popis sa dá nájsť v sekcii Prílohy (Príloha \ref{in:ten})), súbor ku každému vstupnému neurónu vydá aj očakávaný výstup (1?, 2?) a pre predikovanú časť dodá aj kurzy stávkových kancelárií na daný výsledok (1B, 2B).
Poradie a význam stĺpcov je popísaný v sekcii Prílohy (\ref{in:ten}).

Z popisu datasetu (v sekcii \ref{ten}) je vidieť, že hráči v zápasoch sú zoradení tak, že najprv je napísaný víťaz a po ňom porazený. To by nám očividne zamiešalo výsledkami a ak by to sieť zistila, tak by okamžite vypisovala úspešnosť 100\,\%.
Presne z tohto dôvodu robí program \textit{ATPDataMaker.exe} aj randomizovanú výmenu poradia hráčov a v ďalšom priebehu sú hráči rozlišovaní ako hráč 1 a hráč 2.

Cestu na dané súbory potom ako prvé tri parametre (v poradí, v akom sú uvedené v úvode sekcie) predáme programu \textit{ffnnatp.py} alebo \textit{rnnatp.py} podľa toho, či chceme, aby dané údaje vyhodnocovala popredná rekurentná neurónová sieť.
Výsledky vypíše na štandardný výstup a uloží ich aj do logu, ktorý je vo formáte \textit{txt} a ktorého názov pozostáva z typu siete, slova \textit{atp} a časovej známky.



\chapter*{Záver}
\addcontentsline{toc}{chapter}{Záver}

Jedným z dôvodov, prečo ľudia sledujú šport je ten, že šport je nepredvídateľný. Na druhej strane, ľudstvo sa učí od nepamäti získavať kontrolu nad nepredvídateľným a ak sa to nepodarí, tak aspoň to s nejakou určitosťou predvídať (napríklad zatmenie slnka). Vedieť predvídať športové výsledky by nám zaručilo nielen slávu a pozornosť médií, ale mohlo by nám to pred týmto všetkým zaručiť aj bohatstvo, pretože stávkové kancelárie dávajú možnosť zarobiť na správnom tipovaní (v našom prípade predpovedaní) výsledkov.

Jedným z relatívne nových spôsobov, ktorý sa začína používať na predpovedanie športových výsledkov je strojové učenie. Jedným z najznámejších a momentálne aj naobľúbenejších typov strojového učenia sú neurónové siete. V tejto práci sme sa zamerali na dva druhy neurónových sietí, a to dopredné a rekurentné neurónové siete a ich úspešnosť pri predikcii výsledkov v troch futbalových ligách a na najvyšších tenisových turnajoch medzi najlepšími hráčmi.

Používali sme len dáta, ktoré vieme vyčítať z výsledkov (v prípade tenisu aj z koncoročných rebríčkov), ale nepodarilo sa nám ani priblížiť výsledkom iných autorov publikujúcich v tejto oblasti, ktorý používali aj fakticky nepodložené abstraktné dáta od expertov v oblasti. V prípade futbalu sme dosiahli úspešnosť od 50,54 -- 56,24 \,\%. V prípade tenisu sme dosiahli úspešnosť okolo 65,5\,\%.

Jednou z odlišností tejto práce od ostatných v oblasti bolo zameranie sa aj na zápasy bez jasného favorita stávkových kancelárií. Aj napriek tomu, že to bol celkom nový prístup, nepodarilo sa nám dosiahnúť požadované výsledky, ktoré by mohli podnietiť väčší výskum týmto smerom. To ale neznamená, že sa lepšie výsledky dosiahnuť nedajú. Znamená to iba, že sa zo získaných dát ukázaným spôsobom nebude dať vytvoriť neurónová ani rekurentná neurónová sieť, ktorá by dosahovala stabilné výsledky a dokázala by pravidelne a dlhodobo poraziť stávkové kancelárie a vykázať teda nejaký zisk.

%%% Seznam použité literatury
\include{literatura}

%%% Obrázky v bakalářské práci
%%% (pokud jich je malé množství, obvykle není třeba seznam uvádět)
\listoffigures

%%% Tabulky v bakalářské práci (opět nemusí být nutné uvádět)
%%% U matematických prací může být lepší přemístit seznam tabulek na začátek práce.
\listoftables

%%% Použité zkratky v bakalářské práci (opět nemusí být nutné uvádět)
%%% U matematických prací může být lepší přemístit seznam zkratek na začátek práce.
\chapwithtoc{Seznam použitých zkratek}

%%% Přílohy k bakalářské práci, existují-li. Každá příloha musí být alespoň jednou
%%% odkazována z vlastního textu práce. Přílohy se číslují.
%%%
%%% Do tištěné verze se spíše hodí přílohy, které lze číst a prohlížet (dodatečné
%%% tabulky a grafy, různé textové doplňky, ukázky výstupů z počítačových programů,
%%% apod.). Do elektronické verze se hodí přílohy, které budou spíše používány
%%% v elektronické podobě než čteny (zdrojové kódy programů, datové soubory,
%%% interaktivní grafy apod.). Elektronické přílohy se nahrávají do SISu a lze
%%% je také do práce vložit na CD/DVD. Povolené formáty souborů specifikuje
%%% opatření rektora č. 72/2017.
\appendix
\chapter{Přílohy}

\section{První příloha}

\openright
\end{document}
