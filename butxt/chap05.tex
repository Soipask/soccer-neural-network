\chapter{Dokumentácia}

Z programátorského hľadiska je práca rozdelená na tri časti. 
Prvú časť predstavuje získavanie výsledkov a kurzov jednotlivých zápasov. 
Druhú časť programu predstavuje transformácia dat na údaje priamo vložiteľné do vstupných neurónov daných neurónových sietí.
Poslednú časť tvorí stavba daného typu neurónovej siete pre daný šport.

Transformačná časť rozdelí data na 3 časti, trénovacie data, trénovacie data (pre optimalizovanie siete používané ako testovacie) a testovacie data.
Je teda zaručené, že žiadna sieť neuvidí testovacie data vopred pred finálnym vyhodnocovaním.

Údaje, ktoré sa objavia vo výstupe sú celková úspešnosť a celkový zisk, úspešnosť a zisk siete pri vyrovnaných zápasoch (a vyrovnaný zápas považujem zápas, kde kurzy na výhru jedného alebo druhého tímu sa líšia najviac o 1) a úspešnosť a zisk siete pri výhradnom tipovaní zápasov, na ktoré máme istú dôveru (od hodnoty, ktorá je počítaná ako rozdiel dvoch najvyšších čísel, ktoré sieť vydá na výstup, jednoducho povedané, rozdiel najpravdepodobnejšej a druhej najpravdepodobnejšej možnosti výsledku zápasu z hľadiska siete).
Táto hodnota dôvery bola tiež vyoptimalizovaná pre každú sieť/program osobitne.

\section{Futbal}

Podmienkou pre futbal je získať všetky zápasy sezóny pre danú ligu. 
Musia byť všetky, pretože v ďalšej časti sa počíta na základe už odohraných zápasov a jeden zápas by mohol skresliť výsledky. 
Jednotlivé ligy boli teda vyberané nielen na základe kvality, ale aj na základe toho, že v pár posledných sezónach sa ani raz nestalo, že zápas musel byť z nejakého hľadiska udelený kontumačne (awarded) jednému z tímov (ako sa napríklad stalo vo francúzskej lige - nejaký zdroj) alebo celá sezóna bola poznačená korupčným škandálom ako v prípade talianskej ligy v sezóne xxxx (zdroj).
Takéto výsledky by nemuseli skresliť stavbu neurónovej siete, ale všeobecne je lepšie, ak sa takýmto situáciám vyhneme.

Údaje o týchto zápasoch sa dajú stiahnuť jednoducho, spustením programu oddscaper.py a zadaním skratky danej ligy pre futbal. (TODO!)
Skratky sú:
\begin{enumerate}
\item ENG - najvyššia anglická liga (Premier League)
\item GER - najvyššia nemecká liga (Bundesliga)
\item SPA - najvyššia španielska liga (La Liga)
\item BUL -
\item QAT -
\end{enumerate}

Program stiahne všetky výsledky a kurzy pre všetky zápasy všetkých kompletných sezón tej-ktorej ligy zo stránky www.oddsportal.com.

Výstup tohto programu predáme programu \textit{DataMaker.exe} (teda ako prvý parameter programu DataMaker.exe je potrebné predať cestu k súboru, ktorý je výstupom súboru oddscraper.py, tento súbor sa volá rovnako ako skratka danej ligy s príponou \textit{.csv}), ktorý je písaný v jazyku C\# a pretransformuje tieto data na vstupné neuróny pre neurónovú sieť. 
Všetkých vstupných neurónov je 44. 
Presné poradie aj popis sa dá nájsť v sekcii Prílohy (Príloha A1).
Tieto údaje boli vybrané špecificky aj s pomocou súvisiacich prác ako údaje, ktoré popisujú stav oboch tímov, ktoré hrajú proti sebe zápas. 
Bonus predstavujú vstupy označené ako skóre, tieto boli vytvorené mnou ako pokus o jednoduchý a presnejší popis formy pomocou jedného údaju namiesto 10.
Ak bude mať teda jeden z týchto neurónov (alebo obe spoločne) úspech, tak bude možné skrátiť počet vstupných neurónov o 10.
Pre upresnenie, výstupom súboru je opäť tabuľka formátu csv, názov je zložený zo skratky pre názov danej ligy a slova \textit{input}.

Cestu na dané súbory potom ako prvé tri parametre (v poradí, v akom sú uvedené v úvode sekcie) predáme programu \textit{ffnnfootball.py} alebo \textit{rnnfootball.py} podľa toho, či chceme, aby dané údaje vyhodnocovala popredná rekurentná neurónová sieť.
Výsledky vypíše na štandardný výstup a uloží ich aj do logu, ktorý pozostáva z typu siete, názvu ligy a časovej známky vo formáte \textit{txt}.

Hodnoty dôvery pre ffnnfootball.py a rnnfootball.py sú \\\\
TODO!!!\\\\
resp.\\\\
TODO!!!\\\\

\section{Tenis}
Údaje o tenisových zápasoch sú predpripravené v súbore atpresults.csv.

Podmienkou pre tenis je získať všetky zápasy každého turnaja ATP typu 500, 1000 a Grand Slam, kde hraje aspoň jeden hráč z Top 100 rebríčka ATP pre danú sezónu.
Dôvodom je fakt, že predikujeme zápasy týchto turnajov medzi hráčmi z Top 100 rebríčka ATP, ale pre týchto hráčov počítame ich momentálnu formu, takže sú pre nás dôležité aj zápasy, ktoré odohrajú proti hráčom mimo Top 100.
Vzhľadom na relatívnu kvalitu turnajov ATP 250 a fakt, že množstvo hráčov z Top 100 sa pravidelne zúčastňuje aj týchto turnajov. 
Čo teda logicky znamená, že niekedy nastúpia dvaja takíto hráči aj proti sebe. 
Takže zoberieme do úvahy aj tieto turnaje (vzájomné zápasy medzi jednotlivými hráčmi na takto ohodnotených turnajoch tiež patria medzi údaje, z ktorých sa stávajú vstupné neuróny (odkaz na prílohu s tenisom (A2?))). 

V tenise sa nemôžeme vyhnúť zápasom, ktoré boli nejakým spôsobom udelené jednému z hráčov, či už bez boja alebo po skreči súpera v priebehu zápasu, pretože zranenia sú súčasťou profesionálneho športu.
Vo futbale sa to obvykle rieši prestriedaním zraneného hráča, v tenise to, prirodzene, nie je možné.
Pre potreby tejto práce máme dve možnosti, buď môžeme tieto zápasy úplne ignorovať alebo ich môžeme započítavať do niektorých oblastí vstupu (ako napríklad forma alebo vzájomné zápasy) a ignorovať inde (predpovedať takéto výsledky je možno nápad pre inú prácu). 
Pre potreby tejto práce budeme tieto zápasy úplne ignorovať, čo znamená, že sa nevyskytnú v trénovacích ani testovacích datach.
Samozrejme, má to svoje výhody aj nevýhody.
Výhodou je, že výsledky budú reálne odzrkadľovať presnosť siete na zápasoch, ktoré sa odohrali a skončili.
Predpovedať zranenie nie je cieľom tejto práce.
Ďalšou výhodou je spravodlivosť oblastí vstupu ako forma a vzájomné zápasy, pretože sa tam berú len zápasy, ktoré sa dohrali dokonca, takže tieto čísla sa v žiadnom okamihu nenafukujú. Napríklad ak hráč natrafí počas turnaja na dvoch/troch súperov, ktorí sa vzdajú, tak by sa mu vo forme ukázali tieto víťazstvá, aj keď to neboli plnohodnotné výhry.
Nevýhodou je, že výsledky nemusia ukazovať reálne výsledky v praxi (pred zápasom nevieme určiť, či sa hráč zraní, ale sieť aj tak vydá svoju predpoveď, aj keď nebola na tieto údaje trénovaná).

Pre tabuľku \textit{atpresults.csv} je postup podobný. 
Túto tabuľku je potrebné predať programu \textit{ATPDataMaker.exe} (opäť ako prvý parameter je potrebené predať cestu k tejto tabuľke).
Program je opäť písaný v jazyku C\# a opäť pretransformuje data na vstupné neuróny pre neurónovú sieť.
Všetkých vstupných údajov (počet stĺpcov tabuľky) je 37.
Ich presné poradie a popis sa dá nájsť v sekcii Prílohy (Príloha A2).

Cestu na dané súbory potom ako prvé tri parametre (v poradí, v akom sú uvedené v úvode sekcie) predáme programu \textit{ffnnatp.py} alebo \textit{rnnatp.py} podľa toho, či chceme, aby dané údaje vyhodnocovala popredná rekurentná neurónová sieť.
Výsledky vypíše na štandardný výstup a uloží ich aj do logu, ktorý je vo formáte \textit{txt} a ktorého názov pozostáva z typu siete, slova \textit{atp} a časovej známky.

Hodnoty dôvery pre ffnnatp.py a rnnatp.py sú \\\\
TODO!!\\\\
resp.\\\\
TODO!!\\\\
