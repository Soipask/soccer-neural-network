\chapter{Dokumentácia} \label{docu}

Z programátorského hľadiska je práca rozdelená na tri časti. 
Prvú časť predstavuje získavanie výsledkov a kurzov jednotlivých zápasov. 
Druhú časť práce predstavuje transformácia dát na údaje priamo použiteľné pri konštrukcii neurónových sietí.
Poslednú časť tvorí stavba daného typu neurónovej siete pre daný šport.

Transformačná časť v prípade tenisu rozdelí dáta na 3 časti, trénovacie dáta, testovacie dáta (pre optimalizovanie siete) a vyhodnocovacie dáta.
V prípade futbalu dáta rozdelí na 2 časti, trénovacie a vyhodnocovacie dáta.
Pre optimalizáciu som použil program podobný programu v poslednej časti, ktorý si trénovacie dáta rozdelíl podľa potreby (konkrétne na prvých 6 celých sezón a prvú polovicu ďalšej a druhú časť tvorí druhá polovicu tejto sezóny, až následne prebiehalo učenie siete).
Je teda zaručené, že žiadna sieť neuvidí vyhodnocovacie dáta vopred pred finálnym vyhodnocovaním.
Vyhodnocovacie dáta sú použité až pre získavanie výsledkov použitých v tejto práci, a teda použili sa až v poslednej fáze.

\section{Futbal}

Údaje o týchto zápasoch sa dajú stiahnuť jednoducho, spustením programu oddscaper.py a zadaním skratky danej ligy pre futbal ako parameter.
Skratky sú:
\begin{enumerate}
\item ENG - najvyššia anglická liga (Premier League)
\item GER - najvyššia nemecká liga (Bundesliga)
\item SPA - najvyššia španielska liga (La Liga)
\end{enumerate}
Program stiahne všetky výsledky a kurzy pre všetky zápasy všetkých kompletných sezón tej-ktorej ligy zo stránky www.oddsportal.com.

Súbor, ktorý vznikne z činnosti tohto programu predáme programu \textit{DataMaker.exe}. 
To znamená, že ako prvý parameter programu \textit{DataMaker.exe} (relatívna cesta tohto programu vzhľadom na prvú zložku \textit{DataMaker} je \textit{DataMaker/bin/Debug}), je potrebné predať názov súboru bez prípony aj s jeho relatívnou cestou, ktorý je výstupom súboru oddscraper.py, (tento súbor sa volá rovnako ako skratka danej ligy s príponou \textit{.csv}), ktorý je písaný v jazyku C\# a pretransformuje tieto dáta na vstupné neuróny pre neurónovú sieť. 
Ak tento súbor neexistuje, je potrebné otvoriť súbor \textit{DataMaker.sln} v prvej zložke \textit{DataMaker} a zdrojové kódy skompilovať v režime \textit{Debug}, aby tento súbor vznikol.
Pre prípad, že pracujeme priamo v zložke, ktorá je elektronickou prílohou k práci, tak relatívna cesta nie je žiadna a stačí ako parameter predať skratku danej ligy (napríklad \textit{DataMaker.exe SPA}).
Všetkých vstupných neurónov je 44. 
Presné poradie aj popis sa dá nájsť v sekcii Prílohy (Príloha \ref{in:foot}).
Program taktiež vydá ako posledné tri stĺpce aj výsledok zápasu ako kategorické hodnoty v poradí domáci, remíza, hostia, kde výsledok, ktorý nastal je ohodnotený 1, zvyšné sú 0. (tiež popísané v Prílohe \ref{in:foot}).
Tieto údaje boli vybrané špecificky aj s pomocou súvisiacich prác ako údaje, ktoré popisujú stav oboch tímov, ktoré hrajú proti sebe zápas (viac v sekcii \ref{foot}). 
Program tiež vytvorí ďalší súbor, ktorý obsahuje testovacie dáta, teda dáta, ktoré sa nevyužívajú pri trénovaní siete, ale len pri vyhodnocovaní výsledkov. 
Tieto dáta sú v rovnakom poradí a musia obsahovať kurzy na dané výsledky a aj výsledok zápasu vo forme troch stĺpcov 
Je to potrebné pre vyhodnocovanie, pretože neurónová sieť bude mať 3 výstupné neuróny v rovnakom poradí a predikovaný výsledok ohodnotí na 1.

Pre upresnenie, súbor vytvorí dva tabuľky opäť s formátom \textit{csv}, názov je zložený zo skratky pre názov danej ligy a slova \textit{input} pre trénovacie dáta, pre testovacie je to skratka danej ligy a slovo \textit{resinput}.

Cestu na dané súbory potom ako prvé dva parametre (v poradí, v akom sú uvedené v úvode kapitoly) predáme programu \textit{ffnnfootball.py} alebo \textit{rnnfootball.py} podľa toho, či chceme, aby dané údaje vyhodnocovala dopredná rekurentná neurónová sieť.

Výpis na štandardný výstup sa mi pre učenie siete nepodarilo potlačiť, takže štandardný výstup bude plný týchto vecí, ale po konci každého cyklu učenia a vyhodnocovania vypíše trénovaciu chybu a trénovaciu úspešnosť, rovnako ako testovaciu úspešnosť, úspešnosť v zápasoch bez favorita a zisk v týchto zápasoch.
Na konci vypíše priemernú trénovaciu a testovaciu úspešnosť, spolu aj s priemernou úspešnosťou v zápasoch bez jasného favorita a zisk pri uzatváraní stávok na tieto zápasy. 
Najdôležitejšie údaje pre nás sa zhrnú do záznamového súboru (\textit{logu}). 
Log obsahuje rôzne údaje od názvu ligy, selektovaných príznakov, cez architektúru siete, až po trénovaciu chybu a trénovaciu a testovaciu úspešnosť, úspešnosť v zápasoch bez favorita a zisk v týchto zápasoch pre každý cyklus učenia a vyhodnocovania a na konci priemerné hodnoty týchto údajov.
Log sa vytvorí v zložke \textit{Logs} a jeho meno bude obsahovať skratku pre ligu, na ktorej prebiehalo učenie, časovú známku vo formáte rok, mesiac, deň, hodina, minúta a sekunda zapnutia finálneho programu a aktivačnú funkciu vo výstupnej vrstve. Na začiatku názov obsahuje skratku \uv{RNN}, ak sa vytvárala rekurentná neurónová sieť. Tento súbor bude vo formáte \textit{txt}.
Napríklad, ak sieť vytvorí textový súbor s názvom \textit{RNNger20190716212031softmax}. Znamená to, že 16.7.2019 o 21:20:31 bola vytvorená neurónová sieť na predikciu výsledkov nemeckej Bundesligy.
Aktivačná funkcia tejto siete na výstupnej vrstve bola softmax.

\section{Tenis}

Údaje o tenisových zápasoch sú predpripravené v súbore \textit{atpresults.csv}.
Súbor \textit{atpranking.csv} je tiež predpripravený a obsahuje ID jednotlivých hráčov, ich mená a ich poradie v koncoročných rebríčkoch hodnotenia ATP za roky 1999--2018 (pár riadkov tejto tabuľky predstavuje obrázok \ref{rank}).
Poradie berieme len ak sa hráč umiestnil na miestach 1--100.
Poradie údajov v tabuľke \textit{atpresults.csv} je popísané v sekcii \ref{ten}.
Na získanie kurzov do tejto tabuľky je ale potrebné spustiť program \textit{atpodds.py} a v zložke, v ktorej sa práve nachádza, je potrebné mať súbory \textit{atpresults.csv} a \textit{atpvenues.csv}, ktorý obshuje len rozličnosti v menách jednotlivých turnajov medzi \textit{atpresults.csv} a stránkou OddsPortal.com, z ktorej sa online sťahujú tieto kurzy.
Mal by som upozorniť, že tento program nie vždy pracoval úplne bezchybne, dôležité je mať kvalitné pripojenie.
Program v zložke, v ktorej sa akutálne nachádza prepíše súbor \textit{atpresults.csv} tak, že už obsahuje dáta v správnom formáte, aby sme ich mohli predať transformačnej vrstve.

Predzápasové kurzy môžu byť prázdne (vyplnené kurzom 0.0), ale len, ak nás na daný zápas nezaujímajú kurzy (zaujímajú nás len pre posledné dva roky, prvý je na testovanie a druhý na vyhodnocovanie).

Dátasety \textit{atpresults.csv} a \textit{atpranking.csv} musia byť v zložke, kde sa nachádza aj prvá zložka \textit{ATPDataMaker}.
Ak sú všetky súbory v zložke rovnako, ako v elektronickej prílohe, tak stačí priamo spustiť program \textit{ATPDataMaker.exe} (ktorého relatívna cesta je \textit{ATPDataMaker/ATPDataMaker/bin/Debug}), ktorý ich pretransformuje na dáta pre vstupné neuróny neurónových sietí. 
Ak tento súbor neexistuje, je potrebné otvoriť súbor \textit{ATPDataMaker.sln} v prvej zložke \textit{ATPDataMaker} a zdrojové kódy skompilovať v režime \textit{Debug}, aby tento súbor vznikol.

Tento súbor vytvorí to zložky, v ktorej sú súbory \textit{atpresults.csv} a 
\textit{atpranking.csv}, tri nové súbory, 
\textit{atp.csv}, 
\textit{atpres.csv} a 
\textit{atpres\_final.csv}. 
Všetky tieto súbory majú údaje v presnom poradí, ako je popísané v prílohe \ref{in:ten}.

Z popisu dátasetu (v sekcii \ref{ten}) je vidieť, že hráči v zápasoch sú zoradení tak, že najprv je napísaný víťaz a po ňom porazený. To by nám očividne zamiešalo výsledkami a ak by to sieť zistila, tak by okamžite vypisovala úspešnosť 100\,\%.
Presne z tohto dôvodu robí program \textit{ATPDataMaker.exe} aj randomizovanú výmenu poradia hráčov a v ďalšom priebehu sú hráči rozlišovaní ako hráč 1 a hráč 2.

Potom môžeme konečne spustiť programy, ktoré vytvárajú neurónové siete.
V zložke, v ktorej sa nachádzajú tieto programy sa musia nachádzať aj tri zložky vytvorené v transformačnej vrstve programom \textit{ATPDataMaker.exe}.
Program textit{ffnnatp.py} vytvorí doprednú neurónovú sieť aj s najlepším modelom získaným pre tenis v sekcii \ref{ffnn:train}.
Program \textit{rnnatp.py} naopak vytvorí rekurentnú neurónovú sieť aj s najlepším modelom získaným v sekcii \ref{rnn:train}.

Výpis na štandardný výstup sa mi pre učenie sietí opäť nepodarilo potlačiť, takže štandardný výstup bude plný týchto výpisov z učenia, ale po konci každého cyklu učenia a vyhodnocovania vypíše trénovaciu chybu a trénovaciu úspešnosť, rovnako ako testovaciu úspešnosť, úspešnosť v zápasoch bez favorita a zisk v týchto zápasoch.
Na konci vypíše priemernú trénovaciu a testovaciu úspešnosť, spolu aj s priemernou úspešnosťou v zápasoch bez jasného favorita a zisk pri uzatváraní stávok na tieto zápasy. 
Najdôležitejšie údaje pre nás sa opäť zhrnú do logu. 
Log obsahuje rôzne údaje od názvu ligy, selektovaných príznakov, cez architektúru siete, až po trénovaciu chybu a trénovaciu a testovaciu úspešnosť, úspešnosť v zápasoch bez favorita a zisk v týchto zápasoch pre každý cyklus učenia a vyhodnocovania a na konci priemerné hodnoty týchto údajov.
Log sa vytvorí v zložke \textit{Logs} a jeho meno bude obsahovať skratku \uv{ATP},  a časovú známku vo formáte rok, mesiac, deň, hodina, minúta a sekunda zapnutia finálneho programu. Na začiatku názov obsahuje skratku \uv{RNN}, ak sa vytvárala rekurentná neurónová sieť podobne ako v minulom odseku.