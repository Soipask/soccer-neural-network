\subsubsection{Súvisiace práce}

V minulosti boli použité rôzne metódy na predikciu športových výsledkov.
V~roku 2005 sa o predpoveď 6 rôznych udalostí týkajúcich sa austrálskej kriketovej ligy a AFL, ligy v austrálskom futbale, pokúsil Bailey. 
Na austrálsky futbal použil data zo zápasov zo 100 sezón odohraných pred rokom 1997 a testoval to na zápasoch od sezóny 1997 do 2003 použitím rôznych modelov lineárnej regresie. 
Dokázal získať presnosť 66.7\% \citep{related:bailey}.

V roku 2006 Joseph, Fenton a Neil vyskúšali viaceré druhy strojového učenia na predikciu výsledkov zápasov tímu Tottenham Hotspur F.C. v najvyššej anglickej futbalovej lige, Premier League, v sezónach 1995/1996 a 1996/1997.
To znamená, že pracovali s datasetom o veľkosti 76 zápasov, z ktorej časť delili na trénovacie a časť na testovacie data. 
Použité metódy zahŕňali expertmi konštruované bayesovské siete, naivný bayesovký klasifikátor, rozhodovacie stromy a k-NN. 
Použili pri tom 30 príznakov, ale až 28 sa viazalo iba na to, či daný hráč nastúpil od začiatku na daný zápas alebo nie, zvyšné dva predstavovali silu súpera a miesto zápasu (či hral predikovaný tím na domácom štadióne alebo nie).
V tomto prípade dosiahli bayesovské siete úspešnosť niečo vyše 59\%, zvyšné metódy sa pohybovali medzi 30 -- 38\% pri disjunktných testovacích a trénovacích datach  \citep{related:joseph}.

V roku 2011 sa dvojica Hucaljuk a Rakipovi{\'c} zameriavala na výber príznakov pri predikcii výsledkov futbalovej Ligy majstrov. 
Pracovali s datami z 96 zápasov, ktoré manuálne ohodnotili podľa 30 príznakov.
Vybrané príznaky predstavovali formu oboch tímov v posledných 6 zápasoch, výsledok posledného vzájomného zápasu týchto dvoch tímov, postavenie v rebríčku, počet zranených hráčov a priemerný počet strelených a inkasovaných gólov.
Neskôr zúžili počet príznakov na 20 a na novovzniknutý dataset bolo aplikovaných 6 rôznych metód strojového učenia, naivný bayesovský klasifikátor, bayesovské siete, LogitBoost, k-NN, Random forest a neurónové siete. Najvyššia dosiahnutá úspešnosť bola 68\%, dosiahli ju neurónové siete \citep{related:hucaljuk}.

V roku 2016 vyskúšali logistickú regresiu na predikciu výsledkov futbalovej Premier League výskumníci okolo Prasetia. 
Stavali na výsledkoch svojich predchodcov a vybrali 4 príznaky, ktoré hrali v predchádzajúcich prácach najväčšiu rolu, konkrétne ohodnotenia pre obranu a útok, pre domácich aj hostí.
Dosiahli úspešnosti v najlepšom prípade 69,5\%.\citep{related:prasetio}

V roku 2014 použili Igiri a Nwachukwu nástroj, ktorý zvaný Rapid Miner. 
Jeho úlohou bolo predikovať výsledky anglickej Premier League. 
Použité techniky boli popredná neurónová sieť a lineárna regresia. 
Neurónová sieť dosiahla úspešnosti 85\%, lineárna regresia 93\%. 
Je potrebné dodať, že neurónová sieť predpovedala všetky typy výsledkov (výhra domácich, prehra, remíza), zatiaľ čo regresia predpovedala len zápasy, ktoré sa v konečnom dôsledku skončili výhrou alebo prehrou domáceho celku, takže celková úspešnosť bola o niečo nižšia. 
Autori dodali, že ak sa predpokladá, že zápas môže skončiť aj remízou, tak neurónové siete mali lepšie výsledky. 
K predikcii použili rôzne príznaky vrátane kurzov, priemerný počet striel, striel na bránu, rohových kopov, ale aj abstraktnejšie príznaky ako ofenzívna/defenzívna sila mužstva a ohodnotenie sily jednotlivých hráčov a kvality manažéra \citep{related:igiri}.

V tom istom roku sa Shin a Gasparyan pokúsili nájsť nové metódy predikcie. 
Navrhli použiť data z videohry FIFA 2015 na predikciu španielskej La Ligy.
Použitie tohto návrhu odôvodnili tým, že vydavatelia videohier v dnešnej dobe pracujú na tom, aby boli ich hry čo možno najreálnejšie.
To sa hlavne týka športových hier, kde je dôležité, aby bol každý hráč ohodnotený čo možno najpresnejšie, aby sa to podobalo realite.
FIFA 2015 používa rôzne atribúty na ohodnotenie hráča ako napríklad zrýchlenie, strely z diaľky alebo reflexy pre post brankára.
Tieto data sa získavajú oveľa jednoduchšie ako z iných zdrojov.
Autori vytvorili dva typy modelov: učenie s učiteľom (supervised learning) a bez učiteľa (unsupervised learning).
Pri učení s učiteľom vytvorili 2 prístupy, reálny prediktor, ktorý využíval reálne data a virtuálny prediktor, ktorý využíval práve data z popísanej videohry.
Obe využívali logistickú regresiu a metódu podporných vektorov (support-vector machine).
Reálny prediktor dosiahol úspešnosť 75\%, virtuálny 80\%, čo podľa autorov dokazuje, že data získané z videohier sa dajú používať aj v reálnom svete.
Učenie bez učiteľa analyzovalo stratégie tímov podľa typov hráčov, ktorí sú v danom tíme pomocou k-means clusteringu. Zistili, že lepšie tímy zvyknú mať útočnejšie stratégie a slabšie tímy dokážu uhrať lepšie výsledky proti silnejším tímom, ak majú defenzívnejšiu stratégiu. \citep{related:shin}. \\
